\documentclass[a4paper,10pt,ngerman]{scrartcl}
\usepackage{babel}
\usepackage[T1]{fontenc}
\usepackage[utf8]{inputenc}
\usepackage{textcomp}
\usepackage[a4paper,margin=2.5cm,footskip=0.5cm]{geometry}

% Die nächsten drei Felder bitte anpassen:
\newcommand{\Aufgabe}{Aufgabe 1: Flohmarkt} % Aufgabennummer und Aufgabennamen angeben
\newcommand{\TeilnahmeId}{55628}       % Teilnahme-Id angeben
\newcommand{\Namen}{Michal Boron} % Namen der Bearbeiter/-innen dieser Aufgabe angeben
 
% Kopf- und Fußzeilen
\usepackage{scrlayer-scrpage, lastpage}
\setkomafont{pageheadfoot}{\large\textrm}
\lohead{\Aufgabe}
\rohead{Teilnahme-Id: \TeilnahmeId}
\cfoot*{\thepage{}/\pageref{LastPage}}

% Position des Titels
\usepackage{titling}
\setlength{\droptitle}{-1.0cm}
\usepackage{seqsplit}
\usepackage{verbatim}

% Für mathematische Befehle und Symbole
\usepackage{amsmath}
\usepackage{amssymb}
\usepackage{amsthm}
\usepackage{mathtools}
%\usepackage{cite}

\usepackage[backend=bibtex]{biblatex}
\addbibresource{flohmarkt.bib}

\usepackage{hyperref}
\hypersetup{
    colorlinks=false,
    linkcolor=blue,
    filecolor=magenta,      
    urlcolor=cyan,
}
% Für Bilder
\usepackage{graphicx}
\graphicspath{ {./tex/images/} }

% Für Algorithmen
\usepackage{algpseudocode}
\usepackage[Algorithmus]{algorithm}
\usepackage{gensymb}
\usepackage{tikz}
\usepackage{pgfplots} 
\usepackage{caption}
\usepackage{subcaption}
\usepackage{array}
\usepackage{makecell}
\usepackage[normalem]{ulem}
\usepackage{longtable}
\usepackage{array,multirow,hhline}

\usepackage[backgroundcolor=lightgray]{todonotes}
\usepackage{minibox}


\usepackage{enumitem}
\usepackage[export]{adjustbox}
\usepackage{csquotes}

\usepackage[backend=bibtex]{biblatex}
\addbibresource{flohmarkt.bib}

% Für Quelltext
\usepackage{listings}
\usepackage{color, colortbl}
\definecolor{mygreen}{rgb}{0,0.6,0}
\definecolor{mygray}{rgb}{0.5,0.5,0.5}
\definecolor{mymauve}{rgb}{0.58,0,0.82}
\definecolor{lightblue}{HTML}{cce6ff}
\definecolor{lightred}{HTML}{ffb3b3}

\lstset{
  keywordstyle=\color{blue},commentstyle=\color{mygreen},
  stringstyle=\color{mymauve},rulecolor=\color{black},
  basicstyle=\footnotesize\ttfamily,numberstyle=\tiny\color{mygray},
  captionpos=b, % sets the caption-position to bottom
  keepspaces=true, % keeps spaces in text
  numbers=left, numbersep=5pt, showspaces=false,showstringspaces=true,
  showtabs=false, stepnumber=2, tabsize=2
}

% Diese beiden Pakete müssen zuletzt geladen werden
\usepackage{hyperref} % Anklickbare Links im Dokument
\usepackage{cleveref}

\newtheorem{lemma}{Lemma}
\newtheorem{definition}{Definition}
\newtheorem{satz}{Satz}
\newtheorem{axiom}{Axiom}
\newtheorem{korollar}{Korollar}

\newcommand{\TODO}[1]{\todo[inline]{TODO: #1}}
\newcommand{\mb}[1]{{\color{red}[MB: #1]}}
\newcommand{\tbf}[1]{\textbf{#1}}
\newcommand{\ttt}[1]{\texttt{#1}}
\newcommand{\fp}{\textsc{Flohmarkt--Problem}}

\usetikzlibrary{fit,backgrounds,positioning}
\tikzset{vertex/.style={circle,draw,minimum size=0.8cm,inner sep=1pt,fill=white}}
\pgfplotsset{compat=newest}

% Daten für die Titelseite
\title{\textbf{\Huge\Aufgabe}}
\author{\LARGE Teilnahme-Id: \LARGE \TeilnahmeId \\\\
	    \LARGE Bearbeiter dieser Aufgabe: \\ 
	    \LARGE \Namen\\\\}
\date{\LARGE April 2021}

\begin{document}

\maketitle
\tableofcontents

\todo[inline,caption={Struktur}]{
  Program: erweitere Zeiträume um Minuten \checkmark
  \begin{itemize}
    \item Definitionen, Modellierung des Problems \checkmark
    \item (Themenbezogene Arbeiten)
    \item Komplexität \checkmark
    \begin{itemize}
      \item Notwendigkeit einer Heuristik \checkmark
    \end{itemize}
    \item Konversion \checkmark
    \item heuristisches Verfahren
    \begin{itemize}
      \item Greedy--Anlegen am Anfang \checkmark
      \item heuristisches Verbesserungsverfahren \checkmark
      \begin{itemize}
        %\item welche Methode? 
        \item hill climbing \checkmark
        %\item simuliertes Abglühen
      \end{itemize}      
    \end{itemize}
    \item Diskussion der Ergebnisse \checkmark
    \begin{itemize}
      \item Grenzen/Mängel der Heuristik 
      \begin{itemize}
        \item was wird nicht erkannt? (edge--cases)
        \item was lässt sich nicht eindeutig ausschließen?
        \item getroffene Annahmen
      \end{itemize}
      \item Qualität der Ergebnisse \checkmark
      \begin{itemize}
        \item Qualität der Ergebnisse am Anfang (Greedy--Verfahren) \checkmark
        \item Qualität bzgl. des großen Flächeninhalt, des Gesamtflächeninhalts aller Rechtecke, \% \checkmark
        \item was und wann kann nicht verbessert werden? (Beispiel 4: 7370) \checkmark
      \end{itemize}      
    \end{itemize}
    \item Laufzeit \checkmark
    \item Umsetzung
      \begin{itemize}
        \item Klasse Rec
        \item Klasse Hole
        \item Klasse Solver
        \item Eingabformat!
      \end{itemize}
    \item Beispiel s. unten
  \end{itemize}
}

\section{Lösungsidee}

\subsection{Formulierung des Problems}\label{sec:definitionen}
Gegeben sei eine Strecke der Länge $N$ und ein Zeitraum von $B$ bis $E$.
Außerdem gegeben sei eine Liste von $Z$ Anmeldungen. 
Die Anmeldugen betreffen die Vermietung eines Teils der Strecke in einer konkreten Zeitspanne.
Jede \textbf{\textit{Anmeldung}} $i$ besteht aus einer Strecke $s_i$ ($0 < s_i \leqslant N$),
einem Mietbeginn $b_i$ ($B \leqslant b_i < E$) und einem Mietende $e_i$ ($b_i < e_i \leqslant E$).
In diesem Problem werden Strecken in volltändigen Metern behandelt 
und alle Zeiten werden in vollständigen Stunden angegeben.
Obwohl $N$ auf 1000 Meter, $B$ auf 8:00 und $E$ auf 18:00 in der ursprünglichen Aufgabe festgelegt sind,
kann die folgende Lösungidee auf beliebige Größen, die die Aufgabenbedingungen erfüllen, übertragen werden.
Das gelieferte Programm kann auch mit unterschiedlichen Werten umgehen.

Die Aufgabe ist ein Optimierungsproblem.
Man soll so eine Teilfolge von $k$ Anmeldugen wählen,
dass alle gewählten Strecken in den angebenen Zeiten vermietet werden können, d.h.,
für jede Anmeldung steht eine freie stetige Strecke der angegbenen Länge 
in der angegebenen Zeitspanne durchgehend zur Verfügung,
und dazu die Mieteinnahmen möglichst hoch sind, wobei der Preis 1 Euro pro Meter pro Stunde beträgt.

Man kann das Problem auf folgende Weise modellieren. 
Wir setzen: $M \coloneqq E - B$.
Wir bilden ein \textit{\textbf{großes Rechteck}} $R$ der Größe $N \times M$.
So kann man analog jede Anmeldung $i$ als ein \textit{\textbf{kleineres Rechteck}}
$r_i$ der Größe $s_i \times m_i$ darstellen, wobei $m_i \coloneqq e_i - b_i$.

So können wir die obige Aufgabe umformulieren:
Wähle so eine Teilfolge $Z'$ von Rechtecken aus $Z$,
die eine Anordnung innerhalb von $R$ bilden,
dass kein Paar der Rechtecke in $Z'$ sich überdeckt und
der Gesamtflächeninhalt aller Rechtecke in $Z'$ maximal ist.
Als Fläche eines kleineren Rechtecks $r_i$ bezeichnen wir das Produkt $m_i \times s_i$.

Genauer gesagt: Jedes Rechteck $r_i$ in $Z'$ besitzt 4 Ecken,
die den folgenden Punkten entsprechen:
$(x_i, b_i), (x_i, e_i), (x_i + s_i, e_i), (x_i + s_i, b_i)$.
Man beachte, dass $b_i$, $e_i$ und $s_i$ fixiert sind. 
So ist die Aufgabe, nur $x_i$ so zu wählen, dass die Bedingungen der Aufgabe erfüllt werden.
Wir können uns dieses Problem so vorstellen, dass die Länge $s_i$ und die Breite $m_i$
jedes Rechtecks $r_i$ sowie seine Anordnung entlang der $y$--Achse fixiert sind,
und wir das Rechteck entlang der $x$--Achse zwischen den $x$--Werten von 0 und $N-s_i$ verschieben können.

Weiter nennen wir unsere Aufgabe das \fp.

\TODO{check, reformulate}

%\subsection{(Themenbezogene Arbeiten)}

\subsection{Komplexität des Problems}\label{sec:komplexitaet}

%Betrachten wir das zu \fp{} zugehörige Entscheidungsproblem:
%Gegeben ein umschließendes Rechteck $R$ und eine Liste $Z$
%von Rechtecken mit fixierten Länge, Breite und Anordnung entlang der $y$--Achse,
%können alle Rechtecke aus $Z$ innerhalb von $R$ so platziert werden, dass
%sie sich paarweise nicht überdecken?

%Offensichtlich kann dieses Problem von einer nichtdeterministischen
%Turingmaschine bezüglich der Eingabelänge in Polynomialzeit gelöst werden.
%Gegeben sei eine Platzierung der Rechtecke aus $Z$ innerhalb von $R$. 
%Man kann leicht einen in Polynomialzeit laufenden Algorithmus entwickeln, der
%anhand der Koordinaten der kleineren Rechtecke überprüft, ob keines
%der Rechtecke über die Grenzen von $R$ hinausreicht und ob
%kein Paar von Rechtecken aus $Z$ sich überdeckt.  
%Somit liegt \fp{} in NP.

In diesem Abschnitt zeigen wir, dass \fp{} ein Optimierungsproblem ist, das NP--schwer ist.
Es ist unmöglich, einen Algorithmus zu bilden, der in Polynomialzeit prüfen würde,
ob ein Ergebnis zum \fp{} optimal ist. 
Um das zu beweisen, zeigen wir, dass das
0/1--Rucksackproblem zum \textsc{Floh\-markt-\-Pro\-blem} reduziert werden kann.
Das bedeutet: Falls das \textsc{Floh\-markt-\-Pro\-blem} in Polynomialzeit gelöst werden kann,
so auch das Rucksackproblem.

Eine Instanz des Rucksackproblems besteht aus einer Rucksackgröße $G$ und aus zwei Listen 
$L_1$ und $L_2$ der Länge $n$. $L_1$ enthält die Größen $g_i$ von $n$ Gegenständen,
$L_2$ enthält ihre Werte $v_i$ $(i=1,...,n)$.
Gesucht ist eine Liste mit $n$ boolschen Werten $a_1, ..., a_n$ mit folgender Eigenschaft:
Die Summe $a_i \times g_i$ $(i=1,...,n)$ ist nicht größer als $G$ und die Summe
$a_i \times v_i$ $(i=1,...,n)$ ist maximal.
Das Optimierungsproblem zum 0/1--Rucksackproblem ist NP--schwer.\cite{garey_johnson_2009}

%Eine Instanz des Rucksackproblems besteht aus einer Liste von Zahlen
%sowie aus einem Rucksack mit einer fixierten Größe.
%Das Problem besteht darin, die Zahlen in den Rucksack 
%so zu packen, dass ihre Summe maximal ist und sie die Größe des Rucksacks nicht überschreitet.


Gegeben sei eine Instanz des Rucksackproblems.
Wir können daraus eine Instanz eines speziellen \fp s
auf folgende Weise generieren. Der Flohmarkt hat die Länge $G$.
Es gibt $n$ Anbieter, die jeweils eine Teillänge $l_i$ des Flohmarkts 
und einen Zeitraum $t_i$ $(i=1,...,n)$ beantragen. 
Wir setzen $l_i\coloneqq g_i$ und $t_i\coloneqq v_i/g_i$.
Alle Anbieter wollen zu demselben Zeitpunkt beginnen. 
Da jeder Mieter an seinem Platz beibt, besteht die Aufgabe darin,
eine solche Auswahl aus der Liste der Anbieter zu finden,
für die die Summe der $l_i$ die Länge $G$ nicht übersteigt und
für die die Summe der Produkte $l_i\times t_i$ maximal wird.
Da $l_i \times t_i = v_i$ ist,
wird die Summe der $l_i\times t_i$ genau dann maximal,
wenn die Summe der $v_i$ maximal wird.
Somit ist dieses spezielle \fp{} äquivalent zum ursprünglichen 0/1--Rucksackproblem.
Wenn wir jede beliebige Instanz des \fp s in Polynomialzeit lösen könnten,
könnten wir auch jedes 0/1--Rucksackproblem in Polynomialzeit lösen.
Somit ist das \fp{} NP--schwer.

%Wir können daraus eine entsprechende Instanz des \textsc{Floh\-markt-\-Pro\-blem}s auf folgende Weise generieren.
%Für jede Zahl im Rucksackproblem bilden wir ein Rechteck der Breite 1 und der Länge, die 
%dieser Zahl entspricht.
%So bilden wir eine Anmeldung im \fp{}, deren Länge der
%Zahl aus dem Rucksackproblem entspricht und der Unterschied zwischen
%dem Beginn und dem Ende der Anmeldung beträgt eine Stunde.
%Außerdem bilden wir ein großes, umschließendes Rechteck, dessen Länge der Größe des Rucksacks entspricht
%und dessen Breite ebenfalls 1 beträgt.
%Somit bilden wir eine Instanz eines Flohmarkts, der eine Stunde dauert und dessen 
%Länge der Größe des Rucksacks entspricht.
%In dem hierdurch entstandenen Problem wählen wir die Rechtecke so, 
%dass sie über die Grenzen des umschließenden Rechtecks nicht hinausreichen
%und der Gesamtflächeninhalt der kleineren Rechtecke maximal ist. 
%Insbesondere beachte man, dass man die kleineren Rechtecke nur entlang der längeren
%Seite des großen Rechtecks bewegen darf.

Korf beschreibt ein sehr ähnliches Problem wie das \fp.\cite{korf} 
In seinem Artikel schildert der Autor das \textit{rectangle packing problem}.
In diesem Problem gibt es eine Menge von kleineren Rechtecken und man soll
ein umschließendes Rechteck mit der minimalen Fläche für alle Rechtecke finden,
sodass die Rechtecke sich nicht überdecken.
In diesem Problem dürfen die Rechtecke nicht gedreht werden.
Der Autor beweist, dass das \textit{rectangle packing problem} NP--vollständig ist.

Da das \fp{} NP--schwer ist, muss man über einen optimalen Algorithmus nachdenken. 
Obwohl unser Problem NP--schwer ist, kann es einen parametrisierten Algorithmus geben,
der dieses Problem in Pseudopolynomialzeit lösen würde.
Das betrifft beispielsweise das Rucksackproblem, das sich durch einen dynamischen 
Ansatz in Pseudopolynomialzeit lösen lässt.\cite{parametrized}

Allerdings, um eine Lösung zu entwickeln, nutzen wir die Hinweise, die Cormen et al. zum Lösen von NP--vollständigen Problemen
in ihrem Buch beschreiben. Grundsätzlich gibt es drei Ansätze zum Lösen eines
solchen Problems.\cite[S.~1106]{cormen}
Erstens, wenn die Eingabe klein genug ist, reicht ein 
Algorithmus mit einer exponentieller Laufzeit aus.
Allerdings lässt sich diese Idee schlecht umsetzen,
sobald die Anzahl der kleineren Rechtecke in der Eingabe sich in der Ordnung von Hunderten befindet.
Die praktische Laufzeit eines exponentiellen Algorithmus wäre in diesem Fall zu groß.
Zweitens beschreiben die Autoren,
dass man bestimmte Grenzfälle ausgliedern kann, die sich in Polynomialzeit lösen lassen.
Diesen Ansatz verwenden wir bei einigen Beispielen und er wird im \cref{sec:diskussion-ergebnisse}
besprochen.
Drittens kann man einen Algorithmus liefern, der nahezu optimale Ergebnisse 
in Polynomialzeit liefert --- eine Heuristik. 


%\TODO{Zeige, das Problem ist NP (überprüfbar in P)\\
%Zeige, das Problem ist NP-schwer: Reduktion zu einem anderen NP-voll. oder NP-schweren Problem.
%Die Reduktionsfunktion muss in Polynomialzeit laufen.\\
%https://stackoverflow.com/questions/4294270/how-to-prove-that-a-problem-is-np-complete}

%\TODO{Notwendigkeit einer Heuristik}

\subsection{Heuristik}
Wir entwickeln ein heuristisches Verfahren, um diesem Problem zu begegnen. 
Wir lassen am Anfang einen Greedy--Algorithmus laufen, um ein Ausgangsergebnis zu erzeugen und
danach entwickeln wir ein deterministisches Greedy--Verbesserungsverfahren, das die Vorgehensweise
eines Bergsteigeralgorithmus (engl. \textit{hill climbing algorithm}) nachahmt. 
Der Verbesserungsalgorithmus optimiert heuristisch das Ausgangsergebnis, 
indem er durch mehrmalige Veränderungen der Platzierung ein lokales Maximum sucht.

\subsubsection{Greedy--Algorithmus}

Wir bilden das große Rechteck $R$ auf ein Koordinatensystem ab.
Die Seite der Länge $N$ verläuft entlang der $x$--Achse und die Seite der Länge
$M$ entlang der $y$--Achse.
Der Wert $B$ (nach der Konversion) wird entsprechend am Punkt $(0, 0)$ abgebildet (s. \cref{fig:rechteck-streifen-platzierung})

Die Größen $N$ und $M$ sind im Programm fest, unabhängig davon, wie viel sie betragen.
Außerdem wurde im \cref{sec:definitionen} festegestellt, dass die Größen $s_i$, $b_i$ und $e_i$
des jeweiligen Rechtecks $r_i$ fest sind und dass wir ein Rechteck $r_i$ nur entlang der $x$--Achse,
also entlang der Seite der Länge $N$ des Rechtecks $R$, bewegen dürfen.
So bietet sich eine Verteilung der kleinere Rechtecke $r_i$ auf kleinere \textit{\textbf{Streifen}}
der Länge $N$ im Rechteck $R$ entlang der $x$--Achse (s. \cref{fig:streifen}).
Die Breite eines solchen Streifen ist äquidistant für alle Streifen und, da
man Stände am Flohmarkt nur zu vollständigen Stunden vermietet, 
beträgt die Breite eines Streifens 1 Stunde.\footnote{Wenn man Zeiten zu
vollständigen Minuten betrachtet,
wird $R$ analog in äquidistante Streifen mit Breite von 1 Minute aufgeteilt.}
Legen wir die folgende Schreibweise fest: Ein Streifen im Rechteck $R$, der die
Stunde $k$ betrifft, also in der Stunde $k$ beginnt und in der Stunde $k+1$ endet, nennen wir $S_k$.


Im Programm sind diese Streifen einfach Listen mit allen kleineren Rechtecken, 
deren Breite $m_i$ sich in diesem Streifen enthält.
Nach der Konversion der Eingabe bilden wir eine Liste $Z$, in der jedes
Rechteck $r_i$ mit seinen genannten Größen $s_i, b_i, e_i$ gespeichert wird.
Dann iterieren wir über jedes Rechteck $m_i$ in $Z$ und fügen wir es in jede Liste $S_j$ fïr alle
$j$ hinzu, die die folgende Bedingung erfüllen: $b_i \leqslant j < e_i$.
Das bedeutet, dass ein Rechteck von $b_i = 1$ (nach Konversion, in vollständigen Stunden)
bis $e_i = 5$ in den folgenden Streifen enthalten wird: $S_1, S_2, S_3, S_4$. Im Streifen $S_5$
wird er nicht enthalten, da die Miete mit dem Anfang der 5. Stunde endet.
Wie Streifen implementiert werden, lesen Sie in der \nameref{sec:umsetzung}.


Nach dieser Vorbereitung der Eingabe erfolgt der Lauf unseres Greedy--Algorithmus, der das
Ausgangsergebnis liefert.
Wir sortieren die Rechtecke $r_i$ in jedem Streifen $S_j$ unabhängig voneinander nach folgenden Kriterien
in dieser Reihenfolge: 
1) fallend nach dem Wert $e_i$,
2) aufsteigend nach dem Wert $b_i$ und
3) fallend nach der Fläche jedes Rechtecks. 
Somit sind die ersten Rechtecke in jeder Liste $S_j$ diejenigen,
deren Wert $e_i$ am größten ist --- oft diejenigen, die am breitesten im Streifen sind.
%\TODO{warum diese Reihenfolge? najpierw załatwaimy od lewej największe, po prawej upychamy najmniejszymi}
Die Reihenfolge wurde so gewählt, damit wir in dieser Reihefolge versuchen,
die Rechtecke aus den Streifen ins große Rechteck $R$ zu platzieren.
Die Idee hinter dieser Platzierung ist, dass wir zuerst die breitesten Rechtecke
„links”, also an niedrigeren $x$--Werten, platzieren, so weit es geht. Dann füllen wir 
die Lücken „rechts“ (an größeren $x$--Werten) mit schmalleren Rechtecken. 
Die grobe Idee ist, dass wir das Rechteck $R$ quasi vom Punkt $(0,0)$ bis zum Punkt
$(N, E)$ mit kleineren Rechtecken füllen.

\begin{figure}[ht]
	\begin{subfigure}[t]{.48 \textwidth}
		\centering
		\input{./tex/tikz/streifen}
		\subcaption{Streifen in einem großen Rechteck $R$.}
		\label{fig:streifen}
		\end{subfigure}
	\begin{subfigure}[t]{.48 \textwidth}
		\centering
		\input{./tex/tikz/greedy-algo}
		\subcaption{Nach der Verarbeitung des Streifens $S_0$. Die Zahlen stellen die Reihenfolge dar,
		in der jedes Rechteck platziert wurde.}
		\label{fig:platzierung}
	\end{subfigure}
	\caption{Die Abbildung des Rechtecks $R$ auf einem Koordinatensystem.
	Die Seiten entlang der $x$--Achse haben die Länge $N$ und die Seiten entlang der
	$y$--Achse haben die Länge $M$.}
	\label{fig:rechteck-streifen-platzierung}
\end{figure}

Wir verarbeiten Streifen für Streifen in der
aufsteigenden Reihenfolge der $y$--Werte, beginned mit dem 0--ten Streifen.
Wir iterieren durch jede Liste $S_j$ und untersuchen jedes Rechteck $r_i$ in diesem
Streifen, ob sein Wert $b_i$ gleich dem Wert $j$ ist, also ob das Rechteck (die Anmeldung) mit dem
aktuellen Zeitpunkt $j$ beginnt.
%\TODO{Das is wichtig, weil ...}
Außerdem prüfen wir, ob ein Rechteck bereits platziert wurde.
Wenn die Werte $b_i$ und $j$ übereinstimmen und $r_i$ noch nicht platziert wurde, 
suchen wir von $x = 0$ bis $x = N$ nach der ersten freien Lücke im Streifen $j$,
die mindestens so groß ist wie die Länge des Rechtecks $s_i$. 
Wenn es so eine Lücke gibt, legen wir $r_i$ ins $R$ und übergehen zum Rechteck $r_{i+1}$.

Nachdem alle Streifen verarbeitet worden sind, ist unser Ausgangsergebnis erzeugt.

In diesem einfachen Algorithmus nutzt man beim Platzieren eines Rechtecks den
Vorteil, dass beim Streifen $j$ nur ein Rechteck $r_i$ platziert werden kann, 
das an diesem Streifen beginnt --- es gilt: $b_i = j$.
Natürlich können andere Rechtecke bereits platziert sein,
aber unsere Vorgehensweise sichert uns, dass es für ein Rechteck $r_i$
genug Platz, also genau: $s_i$, über diesem Rechteck
in den weiteren Streifen $S_{b_i + 1}, S_{b_i + 2}, ..., S_{e_i - 1}$ 
gibt, wenn der Algorithmus entscheidet, dieses Rechteck in $R$ zu platzieren.
Diese Beobachtung ist offensichtlich wahr, da man die Streifen von „unten“ 
(beginnend mit den niedrigeren $y$--Werten im Koordinatensystem) nach „oben“
verarbeitet und bei jedem Streifen $j$ prüft, ob es eine genug große Lücke für ein Rechteck
$r_i$ gibt.
Wenn es eine solche Lücke nicht gibt, bedeutet, dass es im Streifen $j$ und 
möglicherweise in weiteren Streifen $j+1, j+2...$ ein Rechteck gibt, das 
die Platzierung von $j$ unmöglich macht.

Man kann leicht begründen, dass der vorgestellte Algorithmus als Greedy klassifiziert werden
kann.
Mit jedem Schritt des Algorithmus wird die aktuell beste Verbesserungsmöglichkeit gewählt.
Der Algorithmus nutzt die sortierte Reihenfolge der Rechtecke im Streifen, um anhand
des aktuellen Standes im Streifen eine Entscheidung zu treffen, ob ein Rechteck $r_i$ in $R$
platziert werden kann.

Auf der \Cref{fig:platzierung} sieht man den verarbeiteten Streifen $S_0$.
Insbesondere erkennt man gut die Reihenfolge der Sortierkriterien der Rechtecke
im Strefen.




\subsubsection{Heuristisches Verbesserungsverfahren}
Wir probieren, das Ausgangsergebnis zu verbessern.
Bezeichnen wir ab jetzt ein beliebiges Ergebnis, also eine beliebige Anordnung
der kleineren Rechtecke innerhalb des großen Rechtecks $R$, die unser Programm liefert, 
als $C$. Insbesondere nennen wir unser Ausgangsergebnis $C_A$.


Man leicht feststellen, dass man mithilfe des obengenannten Greedy--Algorithmus 
das Ausgangsergebnis nicht optimieren kann. Wir haben begründet, dass dieser Algorithmus
an jeder Stelle stets die aktuell optimale Variante wählt. 
Außerdem dürfen wir diesen Algorithmus nicht nochmal nutzen,
da wir voraussetzen, dass die Streifen in der aufsteigender Reihenfolge ein nach dem anderen
verarbeitet werden. Dann kann es sein, dass es sich eine Lücke zwischen den Punkten $(x_j, j)$ und 
$(x_j + \ell, j)$ der Länge $\ell$ an einer Stelle in einem Streifen $j$ befindet
und dass ein Rechteck $r_i$ mit $s_i < \ell$ theorethisch hineinpassen würde, aber
es ist nicht mehr gesichert, dass es die Lücken direkt darüber in oberen Streifen $j+1, j+2, ...$
geben würde.


Deshalb führen wir ein neues Verfahren ein. 
Sei $C$ eine beliebige Platzierung von Rechtecken innerhalb von $R$.
Nennen wir $C$ \textit{das aktuelle Ergebnis}. 
Die allgemeine Idee des Verbesserungsverfahrens besteht darin,
man findet eine Lücke in einem Streifen 
und man legt ein noch nicht platziertes Rechteck $r$ in die Lücke, gegebenfalls
muss man die Rechtecken, die mit $r$ kollidieren, aus der Platzierung entfernen
und somit entstehen neue Lücken, die mit anderen nicht gelegten Rechtecken gefüllt werden
können.
So kommt man auf ein neues Ergebnis $C'$.
Wir vergleichen die Gesamtflächeninhalte der kleineren Rechtecke innerhalb von $R$
der Platzierungen $C$ und $C'$. 
Wenn $C'$ größer ist als $C$, wird $C'$ das aktuelle Ergebnis und der Vorgang
wiederholt sich. Somit stellt dieses Verfahren das
heuristisches Verfahren, das als ein Bergsteigeralgorithmus klassifiziert werden kann.


\subsection{Diskussion der Ergebnisse}

\subsubsection{Grenzen der Heuristik}\label{sec:diskussion-grenzen}
Im \cref{sec:komplexitaet} wurde bewiesen, dass das \fp{} NP--schwer ist.
Es gibt faktoriell viele möglichen Anordnungen der Rechtecke.
Für $|Z|$ in der Ordnung von ca. 700 wird eine Brute--Force--Lösung deshalb nicht in einer
akzeptablen Zeit gelöst.
Die Komplexität des \fp{}s setzt Schranken auf die optimale Lösung des Problems.
Man kann keinen in Polynomialzeit laufenden Algorithmus entwickeln, der jede Instanz dieses
Problems lösen würde.
Das ist der Hauptgrund, warum wir eine Heurstik verwenden.
Allerdings, da eine Heuristik nur ein Approximationsalgorithmus ist und
nur nahezu optimale Ergebnisse liefert, muss es Kompromisse geben.
Dieser Kompromiss betrifft vor allem die Laufzeit und 
dafür, dass das hier vorgestellte Verfahren in Polynomialzeit läuft, 
trifft das Programm an vielen Stellen vereinfachte Entscheidungen,
die nicht zum optimalen Ergebnis führen können.
In diesem Abschnitt diskutieren wir nur über die Grenzen der beiden Greedy--Algorithmen im Programm 
--- über den Greedy--Algorithmus am Anfang und über das heuristische 
Verbesserungsverfahren ---
und im \cref{sec:diskussion-ergebnisse} besprechen wir die Ergebnisse.


Im Greedy--Algorithmus am Anfang liegt die Schwierigkeit darin, dass
die Platzierung der Rechtecke grundsätzlich von ihrer Reihenfolge 
in Listen $S_j$ abhängt. Diese hängt dann von den Sortierkriterien ab. 
Obwohl dank der gewählten Sortierkriterien optimale oder sehr gute Ergebnisse
bei vielen Beispielen herauskommen, ist das nicht der Fall bei allen Beispielen
(mehr dazu im \cref{sec:diskussion-ergebnisse}).
Auf jeden Fall liefert der Greedy--Algorithmus am Anfang kein optimales Ergebnis zum \nameref{ex:2},
weil dieses Ergebnis im Laufe des Verbesserungsverfahrens optimiert wird.


Im Verbesserungsverfahren wurden mehrere Kompromisse zugunsten der Laufzeit gemacht.
Vor allem liegt die Schwierigkeit im Verfahren selbst --- warum wird ein Verfahren verwendet, das
einen Bergsteigeralgorithmus nachahmt 
und nicht z.\,B. ein Verfahren mit der simulierten Abkühlung oder ein ganz anderer heuristischer Ansatz?
Außerdem liegen die Schwierigkeiten des Verbesserungsverfahrens auch an
den Reihenfolgen der Listen $U_j$ und der Liste $H$.
Zum Sortieren dieser Listen nutzt man auch festgelegte Sortierkriterien,
die nicht zwingend das optimale Ergebnis liefern müssen.
Dazu liegt das Problem auch beim Platzieren eines Rechtecks in eine Lücke. 
Wir entscheiden uns, das Rechteck an die Koordinate $x_2$ der Lücke zu legen.
Man könnte hier eine andere Vorgehensweise anwenden, z. B. man könnte das
Rechteck an die Koordinate $x_1$ orientieren; man könnte auf eine besondere
Weise vorgehen, wenn es Rechtecke gibt, die deutlich kleiner sind als eine Lücke; oder
man könnte die Rechtecke in darüber und darunter stehenden Streifen in Betracht ziehen, 
was im Programm nicht gemacht wird.
Wie beim Greedy--Algorithmus am Anfang, hängt die Reihenfolge der Rechtecke 
beim Ausfüllen der Lücken von der Reihenfolge in den Listen $S_j$ ab. 
Weiterhin ist die Weise, auf die Lücken bestimmt werden, sehr stark vereinfacht. 
Man nimmt keinen Bezug darauf, wie breit eine Lücke ist, also wie viele Streifen sie umfasst.
Das ist auf jeden Fall eine Schwachstelle des Programms.
\subsubsection{Qualität der Ergebnisse}\label{sec:diskussion-ergebnisse}
Im vorangegangenen Abschnitt werden die Grenzen der Heuristik erkannt. 
In diesem Abschnitt diskutieren wir die Qualität der herausgekommenen Ergebnisse.

Zuerst bestimmen wir die Kriterien, unter denen wir ein Ergebnis auswerten:
\begin{itemize}
	\item Das Verhältnis vom Gesamtflächeninhalt der ins große Rechteck gelegten Rechtecke 
	zum Gesamtflächeninhalt aller Rechtecke.
	\item Das Verhältnis vom Gesamtflächeninhalt der ins große Rechteck gelegten Rechtecke 
	zum Flächeninhalt des großen Rechtecks.
	\item Die praktische Laufzeit des Programms für ein Ergebnis.
\end{itemize}

Diese Kriterien wurden in Bezug auf die Aufgabenstellung gewählt,
„um den Organisatoren des Flohmarkts zu helfen.“
Das erste Kriterium gibt den Veranstaltern den Einblick darin,
wie viel sie in Bezug auf die Menge des verfügbaren Geldes verdienen ---
alle Vermieter wollen den Veranstaltern Geld anbieten, aber es hängt von den 
Organisatoren ab, welche Anmeldungen sie ablehnen und welche annehmen.
Das zweite Kriterium gibt den Einblick darin, wie viel Geld die 
Veranstalter verdienen in Bezug auf den verfügbaren Platz. 
Die beiden ersten Kriterien liefern natürlich auch die Erkenntnis über die Verlüste, die 
mit der Auswahl an Anmeldungen verbunden sind.
Das dritte Kriterium spielt für die Veranstalter eine praktische Rolle.
Sehr wenige Personen würde ein Ergebnis interessieren, das vielleicht um ein paar Promile 
besser ist, aber dessen Bestimmung mehrere Stunden (oder Tage!) dauert. 
Deshalb wurde auch die Brute--Force--Lösung ausgeschlossen.

\cref{tab:ergebnisse} stellt die Ergebnisse zu den Beispielen von \hyperref[ex:1]{1} bis \hyperref[ex:7]{7}
aus der BWINF--Webseite dar.
In den ersten zwei Spalten befinden sich entsprechend die Sortierkriterien für Liste
$H$ und die Listen $U_j$. Die Sortierkriterien werden in der \cref{tab:kriterien} erläutert.
Die erste Zeile in der Tabelle stellt die Ergebnisse bevor dem Lauf des Verbesserungsverfahrens dar.
Die Zahlen zwischen den Spalten „Bsp. 1“ und „Bsp. 7“ sind die Ergebnisse in [m $\times$ h]
(nach der Aufgabenstellung: auch in Euro),
die das Programm unter Verwendung von den angegebenen Sortierkriterien liefert.
Dazu ist zu beachten, dass der Flächeninhalt des Rechtecks $R$ in allen abgebildeten Beispielen
10000 beträgt.
 
\input{./tex/other/tab_ergebnisse}
\input{./tex/other/tab_kriterien}

Man kann leicht feststellen, dass der Greedy--Algorithmus sehr gute Ergebnisse liefert, die 
manchmal nur um winzige Prozentpunkte durch den Verbesserungsalgorithmus verbessert werden.

Im Beispiel 1 werden alle Rechtecke bereits beim Lauf des Greedy--Algorihtmus am Anfang platziert.
Wenn man den Gesamtflächeninhalt aller Rechtecke aus diesem
Beispiel ausrechnet, kommt man auf die Zahl 8028.
Das ist also das optimale Ergebnis.
Ebenfalls ist beim Beispiel 6 sehr leicht zu erkennen, dass 
der Gredy-Algorithmus das optimale Ergebnis liefert, da alle 
Rechtecke aus dem Beispiel platziert und die Fläche des großen Rechtecks vollständig bedeckt wurden.

Das Beispiel 3 ist ein sehr interessanter Fall. 
Weder werden alle Rechtecke aus diesem Beispiel platziert (das ist sowieso unmöglich, da der Gesamtflächeninhalt aller Rechtecke 10010 beträgt), noch ist 
die Fläche des großen Rechtecks völlig bedeckt. 
Allerdings, wenn wir die Ergebnisse jedes Streifens einzeln betrachten, stellen wir fest,
dass der Flohmarkt von 11:00 bis 17:00 vollständig durchgehend ausgebucht ist, also beträgt
der Flächeninhalt jedes Streifens von 3 bis 8 (einschließlich) 1000.
Das bedeutet, man kann das Ergebnis für diese Streifen nicht verbessern. 
Man kann auch feststellen, dass alle Rechtecke, die zu den Streifen 0, 1, 2, 9 gehören,
platziert wurden.
Das bedeutet, dass die restliche Fläche
mit keinen Rechtecken bedeckt werden kann. Das bedeutet, dass 8778 das optimale Ergebnis für dieses
Beispiel ist, weil man dieses Ergebnis nicht verbessern kann.

Die Situation mit dem Beispiel 4 sieht ähnlich aus. 
Obwohl nicht alle Rechtecke platziert werden (wieder beträgt der Gesamtflächeninhalt aller
Rechtecke mehr als 10000) und es noch viel freien Platz im großen Rechteck gibt (mehr als ein Viertel des
Flächeninhalts), ist dieses Ergebnis optimal.
Man kann per Hand prüfen, dass jede Kombination mit den zwei nicht gelegten
Rechtecken kein besseres Ergebnis ergibt.

Dann bleiben noch die Ergebnisse zu den Beispielen 2, 5 und 7. 
Da die Anzahl an kleineren Rechtecken zu groß ist, um die Ergebnisse per Hand zu prüfen und
wir keinen Brute--Force--Algorithmus laufen lassen, kann man schwer sagen,
wie weit die Ergebnisse von den Optima abweichen. 
Zur Auswertung dieser Ergebnisse verwenden wir unsere festgelegten Kriterien.

Das Ergebnis zum Beispiel 2 bevor dem Lauf des Greedy-Algorithmus am Anfang 
beträgt 9056. Dieser Gesamtflächeninhalt entspricht 90,5\% des Flächeninhalts des
großen Rechtecks und ebenfalls 90,5\% des Gesamtflächeninhalts aller Rechtecke in diesem Beispiel.
Man beachte, dass der Gesamtflächeninhalt aller Rechtecke 10000 überschreitet.
Durch das Verbesserungsalgorithmus steigt das Anfangsergebnis auf 9077,
also eine Verbesserung um 0,2 Prozentpunkte.
In diesem Beispiel gibt es 603 Rechtecke und das sind Anmeldungen, deren Wert $s_i$ hauptsächlich zwischen 
1 und 6 liegt. Die praktische Laufzeit des Algorihtmus schließt sich im Bereich von 20 Sekunden
auf einem modernen Recher.\footnote{Die genaue Laufzeit ist nicht interessant, ohne dass die 
technischen Spezifikationen des Rechners angegeben werden.}
Der Wert 90,7\% ist akzeptabel in Bezug auf die möglichen Einkommen und auf den verfügbaren Platz
und aus praktischer Sicht ist es schwer, sich ein besseres Ergebnis innerhalb von 20 Sekunden zu wünschen,
solange die Ermittlung eines qualitativen Ergbnises per Hand sehr lange dauern würde.
Es ist anhand der \cref{tab:ergebnisse} festzustellen, dass das Ergebnis sich nach dem Lauf des
Verbesserungsalgorithmus unter Verwendung von beliebigen Kriterien um denselben Wert verbessert.

Das Anfangsergebnis zum Beispiel 7 ist 9959 und beträgt genau 99,59\% des Flächeninhalts
des Rechtecks $R$ und des Gesamtflächeninhalts aller Rechtecke. In diesem Beispiel
gibt es 566 Anmeldungen, davon haben die meisten den Wert $s_i$ im Bereich von 1 bis 6, aber
im Vergleich zum Beispiel 2 gibt Rechtecke, die einen Wert $s_i$ im Bereich 10--40 besitzen.
Wenn man die Sortierkriterien \texttt{greaterHolesSize} und \texttt{smallerSize} wählt,
also diejenigen, die für das Program gewählt wurden und die im \cref{sec:verbesserung} beschrieben werden,
liefert das Verbesserungsverfahren ein Ergebnis um 0,2 Prozentpunkte besser.
Wir stellen fest, die anderen Kombinationen der Sortierkriterien und insbesondere die Kombination
aus der letzen Zeile der Tabelle liefert ein besseres Ergebnis.
Das bedeutet, dass der Wert 9979 auf jeden Fall nicht optimal ist. 
Allgeimein ist der Wert 9979 aus praktischer Sicht völlig akzeptabel.
Es ist kaum möglich, innerhalb von einer Sekunde so einen Wert zu erreichen,
wenn man eine Platzierung der Rechtecke per Hand bestimmt.
Aus praktischer Sicht ist der Wert 9991 auch nicht von einem bedeutenden Unterschied zu 9979 und
sind die beiden Werte sehr nah am Optimum.

Als letztes bleibt das Beispiel 5, das 25 große Rechtecke umfasst, und zu dem 
der Greedy--Algorithmus am Anfang das Ergebnis 7962 liefert. 
Der Gesamtflächeninhalt aller Rechtecke beträgt mehr als 300000 und somit
entspricht das Ergebnis 79,6\% des Flächeninhalts des Rechtecks $R$ und
25,7\% des Gesamtflächeninhalts aller Rechtecke.
Das Verbesserungsverfahren unter Verwendung von den Sortierkriterien \texttt{greaterHolesSize}
liefert einen Wert 8705, also ist das eine Verbesserung um 7,4\% Prozentpunkte.
Hingegen beträgt das verbesserte Ergebnis nur 8599, wenn man 
die Sortierkriterien \texttt{smalllerHolesSize} verwendet. 
Es ist schwer zu beurteilen, wie weit das Ergebnis vom Optimum abweicht. 
Allerdings ist 87\% des Flächeninhalts des großen Rechtecks, also des verfügbaren Platzes,
noch in Ordnung und so ein Wert ist aus praktischer Sicht wünschenswert. 
\TODO{s. TeX-Datei dazu}
%dodać motyw z kwotą wolną od podatku

Die deutlichste Verbesserung im Beispiel 5 liegt zugrunde der Entscheidung,
die Sortierkriterien\break \texttt{greaterHolesSize} und \texttt{smallerSize} zu wählen.
Eventuell könnte man den Algorithmus so entwickeln, dass man ihn 10--mal laufen lässt
und in jedem Lauf andere Kriterien wählt. Dann könnte einfach das Maxmimum aus 
diesen Ergebnissen gezogen werden. In vielen Heuristiken ist es genau die Idee, verschiedene Methoden
zusammen zu verknüpfen.
Bei allen Beispielen außer 2 dauert die Laufzeit des ganzen Programm etwa eine Sekunde,
was praktisch eine sehr gute Laufzeit darstellt, und die Qualität der Ergebnisse 
ist in diesem Zusammenhang sehr gut, wenn man dies mit einer faktoriellen Laufzeit 
eines Brute--Force--Lösung vergleicht. Die durch den Bergsteigerlagorithmus 
gefundenen Maxima liegen höchstwahrscheinlich sehr nah an den Optima.

Zum Vergleich wurde auch ein Ansatz mit simulierten Abglühen und ein Ansatz, in
dem die Reihenfolgen der Rechtecke in den Listen zufällig war, ausprobiert, aber
die Ergebnisse waren allgemein schlechter als die, die der Quasibergsteigeralgorithmus liefert.

\subsection{Laufzeit}
\begin{itemize}
	\item $M$ --- die Breite des großen Rechtecks $R$, die Anzahl der Streifen 
	\item $n$ --- $|Z|$, also die Anzahl der kleineren Rechtecke
\end{itemize}

Die Größe $M$ in der Aufgabe tritt in vollständigen Stunden vor.
Wenn die Eingabe zu Minuten konvertiert wird, wird diese Variable
in vollständigen Minuten betrachtet.
Im \cref{sec:greedy} wird beschrieben,
dass die Breite jedes Streifens 1 Stunde bzw. 1 Minute entspricht.
Somit kann man die Größe $M$ auch als die Anzahl der Streifen betrachten.


\begin{itemize}
	\item Vorbereitung der Eingabe: $O(M \cdot n \log n)$ (worst--case)
	\begin{itemize}
		\item Einlesen aller Rechtecke und Erstellung der Liste $Z$: $O(n)$

		\item Erstellung von Listen \ttt{placed}, \ttt{unusedRectangles} und \ttt{holes}
		für jeden Streifen (s. \nameref{sec:umsetzung}): $O(M)$

		\item Verteilung jedes Rechtecks auf die Listen $S_j$, zu denen sie gehören: $O(M \cdot n)$ (worst--case)\\
		Im schlimmsten Fall gehört jedes Rechteck zu jeder Liste $S_j$, wenn jede Anmeldung
		den ganzen Zeitraum eines Flohmarkts betrifft.
		So muss jedes Rechteck in jede Liste $S_j$ hinzugefügt werden.

		\item Sortierung der Listen $S_j$: $O(M \cdot n \log n)$ (worst--case)\\
		Es gibt $M$ Listen und in jeder Liste kann es im schlimmsten Fall
		alle Rechtecke geben. Die linear--logarithmische Laufzeit
		ist durch das Sortieren verursacht.
	\end{itemize}

	\item Der Greedy--Algorithmus am Anfang: $O(n(n + M \log n))$ (worst--case, amortisierte Laufzeit)\\
	Obwohl man die Funktion zur Verarbeitung eines Streifens $M$--mal 
	laufen lässt, kann eine Laufzeitanalyse pro Lauf dieser Funktion zu pessimistisch sein.
	Es ist unmöglich, dass ein Platz für $n$ Rechtecke in jedem Streifen $M$--mal gesucht wird,
	da wir voraussetzen, dass jedes Rechteck $r_i$ nur im Streifen $b_i$ gelegt werden kann
	und außerdem können $n$ Rechtecke nicht $M$--mal platziert werden, da jedes Rechteck
	nur einmal gelegt werden kann.
	Stattdessen analysieren wir die amortisierte Laufzeit für das Platzieren jedes Rechtecks allein,
	deshalb wird die endliche Laufzeit mit dem Faktor $n$ multipliziert, da man im schlimmsten Fall
	alle $n$ Rechtecke ins $R$ legen muss.

	\begin{itemize}
		\item Das Finden der passenden, freien $x$--Koordinaten: $O(n)$ (worst--case)\\
		Im schlimmsten Fall muss man in einem Streifen über $n-1$ Rechtecke iterieren,
		um einen freien Platz für ein Rechteck zu finden.

		\item Das Finden der genauen Stelle in den restlichen Streifen: $O(M \log n)$ (worst--case)\\
		Nicht in jedem Streifen müssen sich dieselben Rechtecke befinden und ein Rechteck 
		kann zu mehreren Steifen gehören.
		In jedem Streifen muss man die genaue Position zum Platzieren des Rechtecks finden.
		Da man im Programm \ttt{set} verwendet, sind alle Rechtecke im Streifen
		immer in aufsteigender Reihenfolge ihrer Koordinaten $x_i$. 
		So kann ein Rechteck mittels der eingebauten Funktion \ttt{insert} 
		in \ttt{set} einfügt werden. Die Einfügen--Operation erfolgt in logarithmischer Laufzeit
		bezüglich der Anzahl der Rechtecke im Streifen: $O(\log n)$.\footnote{
		\href{https://en.cppreference.com/w/cpp/container/set/insert}{https://en.cppreference.com/w/cpp/container/set/insert}}
		Im schlimmsten Fall gehört ein Rechteck zu allen Streifen, deshalb muss die endliche Laufzeit
		mal $M$ multipliziert werden: $O(M \log n)$.

		%Das erfolgt mittels der eingebauten Funktion \ttt{upper\_bound}, die in C++ in $O(\log n)$ läuft.\footnote{\href{https://en.cppreference.com/w/cpp/algorithm/upper_bound}{https://en.cppreference.com/w/cpp/algorithm/upper\_bound}}.

	\end{itemize}

	\item Ein Lauf des Verbesserungsalgorithmus pro ein Paar Lücke--Rechteck: $O(n(n + M \log n))$ (worst--case, amortisierte Laufzeit)

	\begin{itemize}
		\item Berechnung des Gesamtflächeninhalts aller platzierten Rechtecke: $O(n)$ (worst--case)\\
		Im schlimmsten Fall können alle Rechtecke ins große Rechteck $R$ gelegt werden.

		\item Bestimmung aller nicht gelegten Rechtecke: $O(M \cdot n \log n)$ (worst--case)\\
		Im schlimmsten Fall gibt es ein gelegtes Rechteck, das genauso groß ist wie $R$, und somit
		alle $n-1$ restlichen kleineren Rechtecke zu Listen $U_j$ gehören. 
		Im schlimmsten Fall können alle diese restlichen Rechtecke zu allen Listen $U_j$ gehören.
		Die linear--logarithmische Laufzeit ist mit dem Sortieren der Listen $U_j$ verbunden.

		\item Auffinden aller Lücken: $O(M \cdot n \log n)$ (worst--case)\\
		Im schlimmsten Fall gibt es in jedem Streifen $n+1$ Lücken, wenn kein Paar
		der Rechtecke in demselben Streifen eine gemeinsame Seite hat ---
		dann gibt es Lücken auf beiden Seiten
		jedes Rechtecks. Dazu kann es im schlimmsten Fall $n$ Rechtecke in jedem Streifen geben.
		Die linear--logarithmische Laufzeit ist mit dem Sortieren der Liste $H$ verbunden.

		\item Entfernung aller kollidierenden Rechtecke: $O(M \cdot n)$ (worst--case)\\
		Beim Einfügen eines Rechtecks $r$ in eine Lücke muss man in allen Streifen, zu denen $r$
		gehört, prüfen, ob es Kollisionen gibt.
		Im schlimmsten Fall gehört ein Rechteck zu allen $M$ Streifen und man muss in allen
		Streifen $n-1$ Rechtecke entfernen.

		\item Einfügen neuer Rechtecke in neue Lücken: $O(n(n + M \log n))$
		(worst--case, amortisierte Laufzeit)\\
		In diesem Teil lohnt es sich mehr, eine amortisierte Laufzeitanalyse pro Rechteck durchzuführen.
		Im schlimmsten Fall muss man $n-1$ Rechtecke ins Rechteck $R$ einfügen, deshalb
		wird die endliche Laufzeit mit dem Faktor $n$ multipliziert.
		Im schlimmsten Fall gehört jedes Rechteck zu jedem der $M$ Streifen.
		Beim Einfügen jedes Rechtecks $r$ muss man zuerst die potenziellen Koordinaten
		für $r$ im Streifen $b_r$ finden.
		Dazu kann man im schlimmsten Fall über $n-1$ Rechtecke iterieren: $O(n)$.
		Dann muss man in jedem Streifen, zu dem $r$ gehört, 
		prüfen, ob es genug Platz dafür gibt.
		Das erfolgt mittels der eingebauten Funktion \ttt{upper\_bound},
		die in logarithmischer Zeit bezüglich der Anzahl der Rechtecke im
		Streifen läuft: $O(\log n)$.\footnote{\href{https://en.cppreference.com/w/cpp/algorithm/upper_bound}{https://en.cppreference.com/w/cpp/algorithm/upper\_bound}}
		%Das wird wieder mittels der Funktion \ttt{upper\_bound} ermittelt,
		%die in logarithmischer Zeit läuft.
		Dann wird das Rechteck zu allen Streifen eingefügt, zu denen es gehört.
		Diese Operation erfolgt mithilfe der Funktion \ttt{insert}, die in
		logarithmischer Zeit läuft: $O(M \log n)$.
		Somit ergibt sich pro Rechteck $r$ die Laufzeit: $O(n + M \log n)$

		\item Zurücksetzung der ursprünglichen Platzierung: $O(M \cdot n)$ (worst--case)\\
		Wenn eine Platzierung einen niedrigeren Flächeninhalt besitzt als 
		die usprüngliche Anordnung, wird diese Platzierung vom Algorithmus abgelehnt
		und die ursprünliche Anordnung wird zurückgesetzt.
		Dazu muss man den Inhalt aus allen $M$ Streifen kopieren, wobei es sich in jedem Streifen höchstens 
		$n$ Rechtecke befinden können.
	\end{itemize}

\end{itemize}

Wie man im \cref{algo:verbesserung} erkennt, unterscheidet man zwischen zwei möglichen
Läufen des Verbesserungsalgorithmus in der While--Schleife:
Wenn ein neues Ergebnis angenommen wird und falls nicht.
Wenn ein Ergebnis angenommen wird, muss man zusätzlich die Funktionen zur Bestimmung aller Lücken und aller nicht gelegten
Rechtecke laufen lassen.
Wie wir in den Betrachtungen zur Laufzeitanalyse zum Verbesserungsalgorithmus pro ein Paar Lücke--Rechteck 
sehen, besitzen diese zwei Funktionen niedrigere
Laufzeiten als die Gesamtlaufzeit der restlichen Prozesse, deshalb unterscheiden wir
nicht zwischen den Laufzeiten der Laüfe des While--Schleife, in denen ein Ergebnis 
angenommen wird, und denjenigen, in denen das nicht der Fall ist.
Somit ergibt sich die folgende Laufzeit für ein Paar von einer Lücke $L$ und einem Rechteck $r$,
wobei $r$ in $L$ eingefügt werden soll: $O(n(n + M \log n))$ (worst--case).

Nun bestimmen wir die Anzahl an Paaren Lücke--Rechteck, die vom Verbesserungsalgorithmus
verarbeitet werden. Die Funktion zur Bestimmung aller nicht gelegten Rechtecke kann
höchstens $n-1$ Rechtecke und die Funktion zum Auffinden aller Lücken kann 
höchstens $M \cdot n$ Lücken finden.
Allerdings, wenn $n-1$ Rechtecke nicht gelegt sind, gibt es höchstens nur $2M$ Lücken.
Wenn die Anzahl der Lücken wächst, sinkt die Anzahl der nicht gelegten Rechtecke.
Deshalb gibt es am meisten Möglichkeiten, wenn die beiden Anzahlen halbiert werden.
So gibt es am meisten $O(M \cdot n/2 \cdot n/2) = O(M \cdot n^2)$ Möglichkeiten.

Im schlimmsten Fall muss man alle diesen Kombinationen durchgehen, bis man
bei einem Ergebnis gelangt, das vom Verbesserungsalgorithmus angenommen wird.
So legen wir fest, die Laufzeit des Verbesserungsalgorithmus
für jede Kombination der Lücken und Rechtecke aus einem Paar von Listen $H$ und $U_j$
beträgt im worst--case:
\[
	O(n(n + M \log n) \cdot M \cdot n^2) = O((n^2 + M \cdot n \log n) \cdot M \cdot n^2)
	= O(M \cdot n^4 + M^2 \cdot n^3 \log n).
\]

Es bleibt noch die Abschätzung, wie viel mal man die Listen $H$ und $U_j$ bestimmt.
Allgemein kann man die Laufzeit eines Bergsteigeralgorithmus schwer abschätzen.
Allerdings ist ein Ergebnis auf den Flächeninhalt des großen Rechtecks $R$ beschränkt
und die Einheiten in dieser Aufgabe sind ganzzahlig (vollständige Meter, vollständige Stunden/Minuten).
So können wir feststellen, dass die Ergebnisse des Verbesserungsalgorithmus mit jeder
neuen Bestimmung der Listen $H$ und $U_j$ eine streng monoton wachsende Funktion bilden.
Es gibt also eine feste Anzahl an möglichen Verbesserungen.
Theoretisch könnte es einen Fall geben, in dem das Anfangsergebnis nach dem Lauf
des Greedy--Algorithmus 1 beträgt und in dem dieses Ergebnis mit jeder 
neuen Bestimmung der Listen $H$ und $U_j$ um 1 verbessert wird,
aber diese Situation ist aus praktischer Sicht kaum vorstellbar. 
Deshalb führen wir eine Variable $k$, deren Wert zwischen 1 und $N \times M$ liegt, die
dafür steht, wie oft die Listen $H$ und $U_j$ bestimmt werden müssen.
In den Beispielen \hyperref[ex:1]{1}--\hyperref[ex:7]{7} überschreitet $k$ nicht 10.
So beträgt die finale Laufzeit im worst--case:
\[
	O(k(M \cdot n^4 + M^2 \cdot n^3 \log n)).
\]


\printbibliography

\section{Umsetzung}\label{sec:umsetzung}
\TODO{Eingabeformat von Minuten erwähnen}

\section{Beispiele}
\subsection{Beispiel 1}\label{ex:1}
Der Flächeninhalt des großen Rechtecks: \framebox[1.1\width]{10000 [m $\cdot$ h]}\\
Der Gesamtflächeninhalt aller platzierten Rechtecke: \framebox[1.1\width]{8028 [m $\cdot$ h]}\\
Der Gesamtflächeninhalt aller Rechtecke: \framebox[1.1\width]{8028 [m $\cdot$ h]}
\subsection{Beispiel 2}\label{ex:2}
\subsection{Beispiel 3}\label{ex:3}
\subsection{Beispiel 4}\label{ex:4}
\subsection{Beispiel 5}\label{ex:5}
\subsection{Beispiel 6}\label{ex:6}
\subsection{Beispiel 7}\label{ex:7}
\todo[inline,caption={Beispiele}]{
  Weitere Beispiele:
  \begin{itemize}
    \item s. Abschnitt Konversion
    \item andere Zeiten \checkmark (Bsp 8)
    \item andere Länge \checkmark (Bsp 8)
    \item mehrtägiger Flohmarkt
    \item mehrere Flohmärkte \sout{+ mehrere Flohmärkte mit unterschiedlichen Längen}
    \item unterbrochener Zeitraum \checkmark (Bsp 10; Pause 12-13)
    \item unterbrochene Länge 
    \item Minuten
    \item edge--cases, bei denen der Algorithmus nicht funktioniert
  \end{itemize}
}


\section{Quellcode}
\lstinputlisting[language=C++]{./tex/flohmarkt.m}

\end{document}