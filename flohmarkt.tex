\documentclass[a4paper,10pt,ngerman]{scrartcl}
\usepackage{babel}
\usepackage[T1]{fontenc}
\usepackage[utf8]{inputenc}
\usepackage{textcomp}
\usepackage[a4paper,margin=2.5cm,footskip=0.5cm]{geometry}

% Die nächsten drei Felder bitte anpassen:
\newcommand{\Aufgabe}{Aufgabe 1: Flohmarkt} % Aufgabennummer und Aufgabennamen angeben
\newcommand{\TeilnahmeId}{55628}       % Teilnahme-Id angeben
\newcommand{\Namen}{Michal Boron} % Namen der Bearbeiter/-innen dieser Aufgabe angeben
 
% Kopf- und Fußzeilen
\usepackage{scrlayer-scrpage, lastpage}
\setkomafont{pageheadfoot}{\large\textrm}
\lohead{\Aufgabe}
\rohead{Teilnahme-Id: \TeilnahmeId}
\cfoot*{\thepage{}/\pageref{LastPage}}

% Position des Titels
\usepackage{titling}
\setlength{\droptitle}{-1.0cm}
\usepackage{seqsplit}
\usepackage{verbatim}

% Für mathematische Befehle und Symbole
\usepackage{amsmath}
\usepackage{amssymb}
\usepackage{amsthm}
\usepackage{mathtools}
%\usepackage{cite}

\usepackage[backend=bibtex]{biblatex}
\addbibresource{flohmarkt.bib}

\usepackage{hyperref}
\hypersetup{
    colorlinks=false,
    linkcolor=blue,
    filecolor=magenta,      
    urlcolor=cyan,
}
% Für Bilder
\usepackage{graphicx}
\graphicspath{ {./tex/images/} }

% Für Algorithmen
\usepackage{algpseudocode}
\usepackage{algorithm}
\usepackage{gensymb}
\usepackage{tikz}
\usepackage{caption}
\usepackage{subcaption}
\usepackage{array}
\usepackage{makecell}

\usepackage[backgroundcolor=lightgray]{todonotes}
\usepackage{minibox}


\usepackage{enumitem}
\usepackage[export]{adjustbox}
\usepackage{csquotes}

\usepackage[backend=bibtex]{biblatex}
\addbibresource{spiess.bib}

% Für Quelltext
\usepackage{listings}
\usepackage{color, colortbl}
\definecolor{mygreen}{rgb}{0,0.6,0}
\definecolor{mygray}{rgb}{0.5,0.5,0.5}
\definecolor{mymauve}{rgb}{0.58,0,0.82}
\definecolor{lightblue}{HTML}{cce6ff}
\definecolor{lightred}{HTML}{ffb3b3}

\lstset{
  keywordstyle=\color{blue},commentstyle=\color{mygreen},
  stringstyle=\color{mymauve},rulecolor=\color{black},
  basicstyle=\footnotesize\ttfamily,numberstyle=\tiny\color{mygray},
  captionpos=b, % sets the caption-position to bottom
  keepspaces=true, % keeps spaces in text
  numbers=left, numbersep=5pt, showspaces=false,showstringspaces=true,
  showtabs=false, stepnumber=2, tabsize=2
}

% Diese beiden Pakete müssen zuletzt geladen werden
\usepackage{hyperref} % Anklickbare Links im Dokument
\usepackage{cleveref}

\newtheorem{lemma}{Lemma}
\newtheorem{definition}{Definition}
\newtheorem{satz}{Satz}
\newtheorem{axiom}{Axiom}
\newtheorem{korollar}{Korollar}

\newcommand{\TODO}[1]{\todo[inline]{TODO: #1}}
\newcommand{\mb}[1]{{\color{red}[MB: #1]}}
\newcommand{\tbf}[1]{\textbf{#1}}
\newcommand{\ttt}[1]{\texttt{#1}}
\newcommand{\fp}{\textsc{Flohmarkt--Problem}}

\usetikzlibrary{fit,backgrounds,positioning}
\tikzset{vertex/.style={circle,draw,minimum size=0.8cm,inner sep=1pt,fill=white}}

% Daten für die Titelseite
\title{\textbf{\Huge\Aufgabe}}
\author{\LARGE Teilnahme-Id: \LARGE \TeilnahmeId \\\\
	    \LARGE Bearbeiter dieser Aufgabe: \\ 
	    \LARGE \Namen\\\\}
\date{\LARGE April 2021}

\begin{document}

\maketitle
\tableofcontents

\todo[inline,caption={Struktur}]{
  \begin{itemize}
    \item Definitionen, Modellierung des Problems
    \item (Themenbezogene Arbeiten)
    \item Komplexität
    \begin{itemize}
      \item Notwendigkeit einer Heuristik
    \end{itemize}
    \item heuristisches Verfahren
    \begin{itemize}
      \item Greedy--Anlegen am Anfang
      \item heurostisches Verbesserungsverfahren
      \begin{itemize}
        \item welche Methode? 
        \item hill climbing
        \item simuliertes Abglühen
      \end{itemize}      
    \end{itemize}
    \item Diskussion der Ergebnisse
    \begin{itemize}
      \item Grenzen/Mängel der Heuristik
      \begin{itemize}
        \item was wird nicht erkannt? (edge--cases)
        \item was lässt sich nicht eindeutig ausschließen?
        \item getroffene Annahmen
      \end{itemize}
      \item Qualität der Ergebnisse
      \begin{itemize}
        \item Qualität der Ergebnisse am Anfang (Greedy--Verfahren)
        \item Qualität bzgl. des großen Flächeninhalt, des Gesamtflächeninhalts aller Rechtecke, \%
        \item was und wann kann nicht verbessert werden? (Beispiel 4: 7370)
      \end{itemize}      
    \end{itemize}
    \item Laufzeit
  \end{itemize}
}

\section{Lösungsidee}

\subsection{Formulierung des Problems}\label{sec:definitionen}
Gegeben sei eine Strecke der Länge $N$ und ein Zeitraum von $B$ bis $E$.
Außerdem gegeben sei eine Liste von $Z$ Anmeldungen. 
Die Anmeldugen betreffen die Vermietung eines Teils der Strecke in einer konkreten Zeitspanne.
Jede \textbf{\textit{Anmeldung}} $i$ besteht aus einer Strecke $s_i$ ($0 < s_i \leqslant N$),
einem Mietbeginn $b_i$ ($B \leqslant b_i < E$) und einem Mietende $e_i$ ($b_i < e_i \leqslant E$).
In diesem Problem werden Strecken in volltändigen Metern behandelt 
und alle Zeiten werden in vollständigen Stunden angegeben.
Obwohl $N$ auf 1000 Meter, $B$ auf 8:00 und $E$ auf 18:00 in der ursprünglichen Aufgabe festgelegt sind,
kann die folgende Lösungidee auf beliebige Größen, die die Aufgabenbedingungen erfüllen, übertragen werden.
Das gelieferte Programm kann auch mit unterschiedlichen Werten umgehen.

Die Aufgabe ist ein Optimierungsproblem.
Man soll so eine Teilfolge von $k$ Anmeldugen wählen,
dass alle gewählten Strecken in den angebenen Zeiten vermietet werden können, d.h.,
für jede Anmeldung steht eine freie stetige Strecke der angegbenen Länge 
in der angegebenen Zeitspanne durchgehend zur Verfügung,
und dazu die Mieteinnahmen möglichst hoch sind, wobei der Preis 1 Euro pro Meter pro Stunde beträgt.

Man kann das Problem auf folgende Weise modellieren. 
Wir setzen: $M \coloneqq E - B$.
Wir bilden ein \textit{\textbf{großes Rechteck}} $R$ der Größe $N \times M$.
So kann man analog jede Anmeldung $i$ als ein \textit{\textbf{kleineres Rechteck}}
$r_i$ der Größe $s_i \times m_i$ darstellen, wobei $m_i \coloneqq e_i - b_i$.

So können wir die obige Aufgabe umformulieren:
Wähle so eine Teilfolge $Z'$ von Rechtecken aus $Z$,
die eine Anordnung innerhalb von $R$ bilden,
dass kein Paar der Rechtecke in $Z'$ sich überdeckt und
der Gesamtflächeninhalt aller Rechtecke in $Z'$ maximal ist.
Als Fläche eines kleineren Rechtecks $r_i$ bezeichnen wir das Produkt $m_i \times s_i$.

Genauer gesagt: Jedes Rechteck $r_i$ in $Z'$ besitzt 4 Ecken,
die den folgenden Punkten entsprechen:
$(x_i, b_i), (x_i, e_i), (x_i + s_i, e_i), (x_i + s_i, b_i)$.
Man beachte, dass $b_i$, $e_i$ und $s_i$ fixiert sind. 
So ist die Aufgabe, nur $x_i$ so zu wählen, dass die Bedingungen der Aufgabe erfüllt werden.
Wir können uns dieses Problem so vorstellen, dass die Länge $s_i$ und die Breite $m_i$
jedes Rechtecks $r_i$ sowie seine Anordnung entlang der $y$--Achse fixiert sind,
und wir das Rechteck entlang der $x$--Achse zwischen den $x$--Werten von 0 und $N-s_i$ verschieben können.

Weiter nennen wir unsere Aufgabe das \fp.

\TODO{check, reformulate}

\subsection{(Themenbezogene Arbeiten)}

\subsection{Komplexität des Problems}\label{sec:komplexitaet}

%Betrachten wir das zu \fp{} zugehörige Entscheidungsproblem:
%Gegeben ein umschließendes Rechteck $R$ und eine Liste $Z$
%von Rechtecken mit fixierten Länge, Breite und Anordnung entlang der $y$--Achse,
%können alle Rechtecke aus $Z$ innerhalb von $R$ so platziert werden, dass
%sie sich paarweise nicht überdecken?

%Offensichtlich kann dieses Problem von einer nichtdeterministischen
%Turingmaschine bezüglich der Eingabelänge in Polynomialzeit gelöst werden.
%Gegeben sei eine Platzierung der Rechtecke aus $Z$ innerhalb von $R$. 
%Man kann leicht einen in Polynomialzeit laufenden Algorithmus entwickeln, der
%anhand der Koordinaten der kleineren Rechtecke überprüft, ob keines
%der Rechtecke über die Grenzen von $R$ hinausreicht und ob
%kein Paar von Rechtecken aus $Z$ sich überdeckt.  
%Somit liegt \fp{} in NP.

In diesem Abschnitt zeigen wir, dass \fp{} ein Optimierungsproblem ist, das NP--schwer ist.
Es ist unmöglich, einen Algorithmus zu bilden, der in Polynomialzeit prüfen würde,
ob ein Ergebnis zum \fp{} optimal ist. 
Um das zu beweisen, zeigen wir, dass das
0/1--Rucksackproblem zum \textsc{Floh\-markt-\-Pro\-blem} reduziert werden kann.
Das bedeutet: Falls das \textsc{Floh\-markt-\-Pro\-blem} in Polynomialzeit gelöst werden kann,
so auch das Rucksackproblem.

Eine Instanz des Rucksackproblems besteht aus einer Rucksackgröße $G$ und aus zwei Listen 
$L_1$ und $L_2$ der Länge $n$. $L_1$ enthält die Größen $g_i$ von $n$ Gegenständen,
$L_2$ enthält ihre Werte $v_i$ $(i=1,...,n)$.
Gesucht ist eine Liste mit $n$ boolschen Werten $a_1, ..., a_n$ mit folgender Eigenschaft:
Die Summe $a_i \times g_i$ $(i=1,...,n)$ ist nicht größer als $G$ und die Summe
$a_i \times v_i$ $(i=1,...,n)$ ist maximal.
Das Optimierungsproblem zum 0/1--Rucksackproblem ist NP--schwer.\cite{garey_johnson_2009}

%Eine Instanz des Rucksackproblems besteht aus einer Liste von Zahlen
%sowie aus einem Rucksack mit einer fixierten Größe.
%Das Problem besteht darin, die Zahlen in den Rucksack 
%so zu packen, dass ihre Summe maximal ist und sie die Größe des Rucksacks nicht überschreitet.


Gegeben sei eine Instanz des Rucksackproblems.
Wir können daraus eine Instanz eines speziellen \fp s
auf folgende Weise generieren. Der Flohmarkt hat die Länge $G$.
Es gibt $n$ Anbieter, die jeweils eine Teillänge $l_i$ des Flohmarkts 
und einen Zeitraum $t_i$ $(i=1,...,n)$ beantragen. 
Wir setzen $l_i\coloneqq g_i$ und $t_i\coloneqq v_i/g_i$.
Alle Anbieter wollen zu demselben Zeitpunkt beginnen. 
Da jeder Mieter an seinem Platz beibt, besteht die Aufgabe darin,
eine solche Auswahl aus der Liste der Anbieter zu finden,
für die die Summe der $l_i$ die Länge $G$ nicht übersteigt und
für die die Summe der Produkte $l_i\times t_i$ maximal wird.
Da $l_i \times t_i = v_i$ ist,
wird die Summe der $l_i\times t_i$ genau dann maximal,
wenn die Summe der $v_i$ maximal wird.
Somit ist dieses spezielle \fp{} äquivalent zum ursprünglichen 0/1--Rucksackproblem.
Wenn wir jede beliebige Instanz des \fp s in Polynomialzeit lösen könnten,
könnten wir auch jedes 0/1--Rucksackproblem in Polynomialzeit lösen.
Somit ist das \fp{} NP--schwer.

%Wir können daraus eine entsprechende Instanz des \textsc{Floh\-markt-\-Pro\-blem}s auf folgende Weise generieren.
%Für jede Zahl im Rucksackproblem bilden wir ein Rechteck der Breite 1 und der Länge, die 
%dieser Zahl entspricht.
%So bilden wir eine Anmeldung im \fp{}, deren Länge der
%Zahl aus dem Rucksackproblem entspricht und der Unterschied zwischen
%dem Beginn und dem Ende der Anmeldung beträgt eine Stunde.
%Außerdem bilden wir ein großes, umschließendes Rechteck, dessen Länge der Größe des Rucksacks entspricht
%und dessen Breite ebenfalls 1 beträgt.
%Somit bilden wir eine Instanz eines Flohmarkts, der eine Stunde dauert und dessen 
%Länge der Größe des Rucksacks entspricht.
%In dem hierdurch entstandenen Problem wählen wir die Rechtecke so, 
%dass sie über die Grenzen des umschließenden Rechtecks nicht hinausreichen
%und der Gesamtflächeninhalt der kleineren Rechtecke maximal ist. 
%Insbesondere beachte man, dass man die kleineren Rechtecke nur entlang der längeren
%Seite des großen Rechtecks bewegen darf.

Korf beschreibt ein sehr ähnliches Problem wie das \fp.\cite{korf} 
In seinem Artikel schildert der Autor das \textit{rectangle packing problem}.
In diesem Problem gibt es eine Menge von kleineren Rechtecken und man soll
ein umschließendes Rechteck mit der minimalen Fläche für alle Rechtecke finden,
sodass die Rechtecke sich nicht überdecken.
In diesem Problem dürfen die Rechtecke nicht gedreht werden.
Der Autor beweist, dass das \textit{rectangle packing problem} NP--vollständig ist.

Da das \fp{} NP--schwer ist, muss man über einen optimalen Algorithmus nachdenken. 
Obwohl unser Problem NP--schwer ist, kann es einen parametrisierten Algorithmus geben,
der dieses Problem in Pseudopolynomialzeit lösen würde.
Das betrifft beispielsweise das Rucksackproblem, das sich durch einen dynamischen 
Ansatz in Pseudopolynomialzeit lösen lässt.\cite{parametrized}

Allerdings, um eine Lösung zu entwickeln, nutzen wir die Hinweise, die Cormen et al. zum Lösen von NP--vollständigen Problemen
in ihrem Buch beschreiben. Grundsätzlich gibt es drei Ansätze zum Lösen eines
solchen Problems.\cite[S.~1106]{cormen}
Erstens, wenn die Eingabe klein genug ist, reicht ein 
Algorithmus mit einer exponentieller Laufzeit aus.
Allerdings lässt sich diese Idee schlecht umsetzen,
sobald die Anzahl der kleineren Rechtecke in der Eingabe sich in der Ordnung von Hunderten befindet.
Die praktische Laufzeit eines exponentiellen Algorithmus wäre in diesem Fall zu groß.
Zweitens beschreiben die Autoren,
dass man bestimmte Grenzfälle ausgliedern kann, die sich in Polynomialzeit lösen lassen.
Diesen Ansatz verwenden wir bei einigen Beispielen und er wird im \cref{sec:diskussion-ergebnisse}
besprochen.
Drittens kann man einen Algorithmus liefern, der nahezu optimale Ergebnisse 
in Polynomialzeit liefert --- eine Heuristik. 


%\TODO{Zeige, das Problem ist NP (überprüfbar in P)\\
%Zeige, das Problem ist NP-schwer: Reduktion zu einem anderen NP-voll. oder NP-schweren Problem.
%Die Reduktionsfunktion muss in Polynomialzeit laufen.\\
%https://stackoverflow.com/questions/4294270/how-to-prove-that-a-problem-is-np-complete}

%\TODO{Notwendigkeit einer Heuristik}



\subsection{Heuristik}
\subsubsection{Ausgangspunkt}
\subsubsection{Heuristisches Verbesserungsverfahren}

\subsection{Grenzen der Heuristik}

\subsection{Laufzeit}




\section{Umsetzung}\label{sec:umsetzung}

\section{Beispiele}

\section{Quellcode}
\lstinputlisting[language=C++]{./tex/flohmarkt.m}

\end{document}