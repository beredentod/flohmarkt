\subsection{Klasse \ttt{Rec}}
Rechtecke werden im Programm als Objekte der Klasse \ttt{Rec} dargestellt.
Diese Klasse besitzt 5 Attribute.
Es gibt: \ttt{b}, \ttt{e} und \ttt{size},
die den Werten $b_i$, $e_i$ und $s_i$ eines Rechtecks $r_i$ entsprechen.
Außerdem gibt es die Attribute \ttt{x1} und \ttt{x2},
die den $x$--Koordinaten eines Rechtecks entsprechen.
Wenn ein Rechteck gelegt wird, werden die beiden Koordinaten festgelegt und 
es gilt dann immer: $\ttt{x1} + \ttt{size} = \ttt{x2}$. 
Wenn ein Rechteck nicht platziert ist, sind die beiden Variablen auf $-1$ gesetzt.

Außer den standardmäßigen Getter-- und Setter--Methoden gibt es auch eine Methode
\ttt{getArea()}, die den Flächeninhalt des Rechtecks ausgibt, indem $(\ttt{e}-\ttt{b})\times \ttt{m}$
ausgerechnet wird. 

\subsection{Klasse \ttt{Hole}}
Die Lücken stellt man als Objekte der Klasse \ttt{Hole} dar. 
In dieser Klasse gibt es 3 Attribute: \ttt{x1}, \ttt{x2} und \ttt{stripe}.
Sie stellen die Koordinaten $x_1$ und $x_2$ einer Lücke dar und
\ttt{stripe} ist der Index des Streifens, in dem die Lücke auftritt.


\subsection{Hilfsfunktionen}
Es gibt drei Hilfsfunktionen, die die Eingabe verarbeiten und die zu keiner Klasse gehören.

Es gibt eine Funktion \ttt{timeToMinutes()}, die zur Umwandlung der Zeitangaben dient.
Sie nimmt einen String als Argument, das entweder im Format \ttt{"H"} oder \ttt{"H:MM"} ist.
\ttt{H} ist dabei ein beliebiger Integer, der
Stunden angibt, und \ttt{MM} ist eine zweistellige Zahl, die Minuten angibt. 
Wenn die Eingabe Minuten enthält, wird die als String eingegebene Zeit zu Minuten umgewandelt,
indem \ttt{H} mal 60 multipliziert wird. Sonst erfolgt keine Konversion und \ttt{H} wird nur 
zu einem Integer umgewandelt.
Ausgegeben wird ein Paar bestehend aus der umgewandelten Zeit und einem Integer, in dem 
das Eingabeformat kodiert wird:\break 0 steht für Stunden und 1 für Minuten.

Die Funktion \ttt{processInput()} verarbeitet den String mit den Öffnungszeiten 
des Flohmarkts.
Diese Funktion nimmt als Argument einen String, der aus $2k$ Zahlen besteht.
Die Funktion teilt den String in eigenständige Uhrzeiten auf.
Jede Uhrzeit wird mithilfe der Funktion \ttt{timeToMinutes()} zu Stunden bzw. zu Minuten umgewandelt.
Dann wird diese Folge von Zahlen als ein \ttt{vector<int>} mit dem Eingabeformat (wie oben) ausgegeben.

Es gibt noch die Funktion \ttt{calculateRecess()}. 
Diese Funktion nimmt als Argument einen \ttt{vector} von Integers, in dem
die Uhrzeiten gespeichert sind.
In diesem \ttt{vector} steht jede Zahl mit einem geraden bzw. ungeraden Index für die Öffnungs-- bzw. Schließungszeit.\footnote{Indexierung ab 0.} 
Ausgegeben wird ein Integer --- die Summe der Unterschiede zwischen jedem $k$--ten Schließungs--
und jedem $k+1$--ten Öffnungszeitpunkt.
Diese Summe ist wichtig, weil man die Schließungszeit des Flohmarkts vom Endergebnis abziehen soll,
wenn der Zeitraum des Flohmarkts unterbrochen ist.