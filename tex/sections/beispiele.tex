\section{Beispiele}
Der Übersichtlichkeit halber befinden sich die genauen Standorte zu Anmeldungen
zu jedem Beispiel in den angehängten csv--Dateien.
In den Tabellen entsprechen die Werte $x_1$ und $x_2$ den Werten 
$x_r$ und $x_r + s_r$ jedes Rechtecks $r$.

\subsection{Beispiel 1}\label{ex:1}
Der Flächeninhalt des großen Rechtecks: \framebox[1.1\width]{10000 [m $\cdot$ h]}\\
Der Gesamtflächeninhalt aller platzierten Rechtecke: \framebox[1.1\width]{8028 [m $\cdot$ h]}\\
Der Gesamtflächeninhalt aller Rechtecke: \framebox[1.1\width]{8028 [m $\cdot$ h]}
\subsection{Beispiel 2}\label{ex:2}
Der Flächeninhalt des großen Rechtecks: \framebox[1.1\width]{10000 [m $\cdot$ h]}\\
Der Gesamtflächeninhalt aller platzierten Rechtecke: \framebox[1.1\width]{9077 [m $\cdot$ h]}\\
Der Gesamtflächeninhalt aller Rechtecke: \framebox[1.1\width]{10002 [m $\cdot$ h]}
\subsection{Beispiel 3}\label{ex:3}
Der Flächeninhalt des großen Rechtecks: \framebox[1.1\width]{10000 [m $\cdot$ h]}\\
Der Gesamtflächeninhalt aller platzierten Rechtecke: \framebox[1.1\width]{8778 [m $\cdot$ h]}\\
Der Gesamtflächeninhalt aller Rechtecke: \framebox[1.1\width]{10010 [m $\cdot$ h]}
\subsection{Beispiel 4}\label{ex:4}
Der Flächeninhalt des großen Rechtecks: \framebox[1.1\width]{10000 [m $\cdot$ h]}\\
Der Gesamtflächeninhalt aller platzierten Rechtecke: \framebox[1.1\width]{7370 [m $\cdot$ h]}\\
Der Gesamtflächeninhalt aller Rechtecke: \framebox[1.1\width]{10534 [m $\cdot$ h]}
\subsection{Beispiel 5}\label{ex:5}
Der Flächeninhalt des großen Rechtecks: \framebox[1.1\width]{10000 [m $\cdot$ h]}\\
Der Gesamtflächeninhalt aller platzierten Rechtecke: \framebox[1.1\width]{8705 [m $\cdot$ h]}\\
Der Gesamtflächeninhalt aller Rechtecke: \framebox[1.1\width]{30940 [m $\cdot$ h]}
\subsection{Beispiel 6}\label{ex:6}
Der Flächeninhalt des großen Rechtecks: \framebox[1.1\width]{10000 [m $\cdot$ h]}\\
Der Gesamtflächeninhalt aller platzierten Rechtecke: \framebox[1.1\width]{10000 [m $\cdot$ h]}\\
Der Gesamtflächeninhalt aller Rechtecke: \framebox[1.1\width]{10000 [m $\cdot$ h]}
\subsection{Beispiel 7}\label{ex:7}
Der Flächeninhalt des großen Rechtecks: \framebox[1.1\width]{10000 [m $\cdot$ h]}\\
Der Gesamtflächeninhalt aller platzierten Rechtecke: \framebox[1.1\width]{9979 [m $\cdot$ h]}\\
Der Gesamtflächeninhalt aller Rechtecke: \framebox[1.1\width]{10000 [m $\cdot$ h]}

\todo[inline,caption={Beispiele}]{
  Weitere Beispiele:
  \begin{itemize}
    \item s. Abschnitt Konversion
    \item andere Zeiten \checkmark (Bsp 8)
    \item andere Länge \checkmark (Bsp 8)
    \item mehrtägiger Flohmarkt
    \item mehrere Flohmärkte \sout{+ mehrere Flohmärkte mit unterschiedlichen Längen}
    \item unterbrochener Zeitraum \checkmark (Bsp 10; Pause 12-13)
    \item unterbrochene Länge 
    \item Minuten
    \item edge--cases, bei denen der Algorithmus nicht funktioniert
  \end{itemize}
}
