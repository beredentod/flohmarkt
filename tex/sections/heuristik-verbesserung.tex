\subsubsection{Heuristisches Verbesserungsverfahren}
Man kann leicht feststellen, dass, wenn alle Rechtecke aus $Z$ im Laufe des
Greedy--Algorithmus in $R$ platziert wurden oder wenn die ganze Fläche von $R$ bedeckt wurde, 
das Problem für diese Eingabe optimal gelöst wurde.
Allerdings lässt sich nicht nachweisen, dass der vorgestellten Algorithmus 
stets eine optimale Platzierung liefert.
Hingegegen kann man sogar festellen, dass es bessere Ergebnisse gibt als die,
die am Anfang geliefert werden. 


Wir probieren, das Ausgangsergebnis heuristisch zu verbessern.
Bezeichnen wir ab jetzt ein beliebiges Ergebnis, also eine beliebige Anordnung
der kleineren Rechtecke innerhalb des großen Rechtecks $R$, die unser Programm liefert, 
als $C$. Insbesondere nennen wir unser Ausgangsergebnis $C_A$.


Offensischtlich kann man mithilfe des obengenannten Greedy--Algorithmus 
das Ausgangsergebnis nicht optimieren. Wir haben begründet, dass dieser Algorithmus
an jeder Stelle stets die aktuell optimale Variante wählt. 
Außerdem dürfen wir diesen Algorithmus nicht nochmal nutzen,
da wir voraussetzen, dass die Streifen in der aufsteigender Reihenfolge ein nach dem anderen
verarbeitet werden. Dann kann es sein, dass es sich eine Lücke zwischen den Punkten $(x_j, j)$ und 
$(x_j + \ell, j)$ der Länge $\ell$ an einer Stelle in einem Streifen $j$ befindet
und dass ein Rechteck $r_i$ mit $s_i < \ell$ theorethisch hineinpassen würde, aber
es ist nicht mehr gesichert, dass es die Lücken direkt darüber in oberen Streifen $j+1, j+2, ...$
geben würde.


Deshalb führen wir ein neues Verfahren ein. 
Sei $C$ eine beliebige Platzierung von Rechtecken innerhalb von $R$.
Nennen wir $C$ \textit{das aktuelle Ergebnis}. 
Die allgemeine Idee des Verbesserungsverfahrens besteht darin,
man führt zu einer Veränderung (Mutation) in $C$, d.h., 
ein nicht platziertes Rechteck $r$ wird in $R$ platziert und es müssen ggf. 
Rechtecke aus $R$ entfernt werden, die mit $r$ kollidieren.
So kommt man auf einen neuen Zustand, eine neue Platzierung der Rechtecke $C'$.
Es wird dann überprüft, ob der Gesamtflächeninhalt aller platzierten Rechtecke in der Platzierung $C'$
größer ist als der in der Platzierung $C$.
Wenn ja, wird $C'$ das aktuelle Ergebnis und der Vorgang wiederholt sich,
bis es noch möglich ein Zustand $C$ zu weiter verändern.
Wir können diesen Ansatz als einen Bersteigeralgorithmus klassifizieren.


Wie kommt es zur Veränderung der Platzierung und wann bestimmen wir,
dass es unmöglich ist, einen Zustand weiter zu verändern?
Unser Verbesserungsverfahren arbeitet mit Lücken, die 
nach der Platzierung $C_A$ entstehen.
Die Idee ist, man findet eine Lücke in einem Streifen 
und man legt ein noch nicht platziertes Rechteck $r$ in die Lücke,
ggf. muss man die Rechtecke,
die mit $r$ kollidieren, aus der Platzierung entfernen
und somit entstehen neue Lücken,
die mit anderen nicht gelegten Rechtecken gefüllt werden können.
So kommt man auf ein neues Ergebnis.
Man hört auf, wenn es keine mehr Lücken gibt, für die ein Rechteck zum Platzieren zur Verfügung steht.


Zuerst muss man die nicht platzierten Rechtecke für jeden Streifen bestimmen. 
So legen wir für jeden Streifen $j$ eine Liste $U_j$ fest, in der sich alle 
Rechtecke aus $j$ befinden, die nicht platziert wurden.
Die Liste $U_j$ muss man auf eine Weise sortieren.
Jedes Rechteck $r_i$ in jeder solchen Liste ordnen wir nach diesen Kriterien:
1) aufsteigend nach der Länge $s_i$ und 2) aufsteigend nach dem Beginn $b_i$.
Die Entscheidung, diese Sortierkriterien zu wählen, ergibt sich experimentell 
und wird im \cref{sec:diskussion-ergebnisse} beschrieben.


Danach muss man die Lücken in einer Platzierung in jedem Streifen finden.
So legen wir eine Liste $H_j$ für jeden Streifen $j$ fest, in der sich alle
Lücken aus diesem Streifen befinden. Man findet sie, indem
man durch jedes im Streifen $j$ platziertes Rechteck $r_i$ iteriert und jeweils überprüft,
ob der Wert $x_i + s_i$ gleich dem Wert $x_{i+1}$ ist. Wenn nicht, gibt es eine Lücke
zwischen den Rechtecken $r_i$ und $r_{i+1}$.
Dazu muss man auch das erste Rechteck untersuchen, ob die Koordinate $x_0$ des Rechtecks $r_0$
dem Wert 0 entspricht bzw. ob die Koordinate $x_l + s_l$ des letzten Rechtecks $r_l$ in diesem Streifen
mit dem Wert $N$ übereinstimmt.
Wenn nicht, enstehen auch Lücken zwischen den Wänden des großen Rechtecks $R$.  

Danach werden alle Listen $H_j$ zu einer Liste $H$ zusammengebracht.
Diese Liste muss auch auf eine Weise geordnet werden. 
Experimentell ergeben sich die folgenden Sortierkriterien für jede Lücke $L_i$: 
1) fallend nach der Größe der Lücke $L_i$ und 2) aufsteigend nach dem
Index des Streifens. Diese Entscheidung wird ebenfalls im \cref{sec:diskussion-ergebnisse}
diskutiert.

\begin{algorithm}[h]
\caption{Das heuristische Verbesserungsvefahren}
\label{algo:verbesserung}
\flushleft
\textbf{Eingabe:} $R$ --- das große Rechteck, $Z$ --- die Liste mit allen kleineren Rechtecken.
\begin{algorithmic}[1]
\State $C_A \gets$ Greedy($R, Z$)
\State $H \gets$ getAllHoles$(C_A)$
\State $U \gets$ getRecs$(C_A)$
\State $C \gets C_A$
\State $itH \gets H.begin$
\State $j \gets itH.$Streifen\Comment{$j$ ist der Index des Streifens, in dem sich die Lücke zum Iterator $itH$ befindet}
\State $itR \gets U_j.begin$
\While{$itH \neq H.end$}\label{line:abbruch}
	\State $\Sigma_C \gets |C|$ \Comment{der Gesamtflächeninhalt aller platzierten Rechtecke in $C$}
	\State $C' \gets$ next$(C, itH, itR)$
	\State $itR \gets itR + 1$
	\State $\Sigma_{C'} \gets |C'|$
	\If {$\Sigma_{C'} > \Sigma_C$} \label{line:vergleich}
		\State $H \gets$ getAllHoles$(C')$
		\State $U \gets$ getRecs$(C')$
		\State $itH \gets H.begin$
		\State $k \gets itH.$Streifen
		\State $itR \gets U_j.begin$
		\State $C \gets C'$
	\EndIf
\EndWhile
\end{algorithmic}
\end{algorithm}

Der \cref{algo:verbesserung} zeigt eine vereinfachte Vorgehensweise des Verbesserungsverfahrens.
Die Funktion $getAllHoles(C)$ findet alle Lücken in allen Streifen im Rechteck $R$ in der
aktuellen Platzierung $C$ und bestimmt die Liste $H$.
Die Funktion $getRecs(C)$ findet alle nicht platzierten Rechtecke und verteilt sie auf
die Listen $U_j$ für jeden Striefen $j$.
Die Funktion $next(C, itH, itR)$ in diesem Algorithmus besteht selbst aus drei Funktionen.

Die erste Funktion bestimmt die nächste Lücke, in die ein nicht platziertes Rechteck 
eingefügt wird. Es gibt einen Iterator $itH$, der am Anfang am Beginn der Liste $H$ gestezt wird.
Im Laufe des Algorithmus bewegt sich der Iterator und zeigt auf nächste Lücken,
Eine Lücke bezeichnen wir als \textit{geeignet}, wenn sie sich in einem
Streifen $j$ befindet, zu dem die entsprechende Liste $U_j$ nicht leer ist, d.h., 
es mindestens ein Rechteck gibt, das in diese Lücke eingefügt werden kann.
Somit bewegt sich der Iterator in der Liste $H$ und, wenn er auf eine geeignete Lücke 
stößt, wird diese Lücke durch die folgenden zwei Funktionen bearbeitet.
Insbesondere beachte man, dass die While--Schleife in \cref{line:abbruch} abbricht, wenn
der Iterator $itH$ zum Ende der Liste $H$ gelangt, also dann,
wenn es keine geeigneten Lücken mehr gibt. 

Die nächste Funktion wählt ein Rechteck $r$, das in eine gewählte geeignete Lücke $L$
eingefügt wird. Da die gewählte Lücke sich in einem Streifen $j$ befindet, stammt $r$
aus der Liste $U_j$.
Es gibt auch einen internen Iterator $itR$ für die Liste $U_j$,
der bei jeder neuen Lücke ans Anfang der Liste gesetzt wird.
Wenn ein Rechteck $r$ in einem Lauf $t$ der While--Schleife in $L$ eingefügt wird,
wird $itR$ danach inkrementiert und im darauf folgenden Lauf der 
Schleife $t+1$ wird ein unterschiedliches Rechteck in die Lücke $L$ gelegt.
Wenn der Iterator $itR$ bis zum Ende der Liste $U_j$ gelangt, wird der Iterator $itH$
in der Liste $H$ inkrementiert und somit eine neue 
geeignete Lücke gesucht.

Die letzte Funktion nimmt ein unter dem Iterator $itR$ stehendes Rechteck $r$
und legt es in die unter dem Iterator $itH$ stehende Lücke $L$.
Diese Funktion bereitet eine neue Platzierung $C'$ vor.
Seien die Koordinaten der Lücke $x_1$ und $x_2$ und der Streifen, in dem sich $L$ befindet, sei $j$. 
Seien $x_r$ und $x_r + s_r$ die $x$--Koordinaten von $r$.
Das Rechteck $r$ wird so gelegt, dass $x_2 = x_r + s_r$. Auf diese Weise beträgt
der Wert $x_r \coloneqq x_2 - s_r$.\footnote{Die Situation, in der $x_r < 0$ gilt, wird schon in der
zweiten Funktion dadurch ausgeschlossen, dass der Iterator $itR$ inkrementiert wird und das nächste
Rechteck gewählt wird.}
Selbstverstädnlich kann an dieser Stelle zu Kollisionen kommen --- Rechtecke können sich überdecken.
Vor dem Platzieren prüft man nicht, ob es durchgehend eine Lücke zwischen $x_r$ und $x_r + s_r$ 
in allen Streifen $k$ gibt, wobei $b_r \leqslant k < e_r$.
Deshalb entfernt man nun alle Rechtecke, die mit $r$ kollidieren, also all diejnigen, die 
zumindest zum Teil zwischen $x_r$ und $x_r + s_r$ in allen Streifen $k$ liegen.
So entstehen auch neue Lücken, deshalb versuchen wir an dieser Stelle 
die Lücken mit anderen Recktecken zu füllen.
Dazu verarbeiten wir alle Streifen $k$ ($b_r \leqslant k < e_r$), indem wir
jedes in $C'$ noch nicht gelegtes Rechteck $r_i$
in jeder Liste $S_k$ (in der ursrpünglichen Reihenfolge) untersuchen.
Wie beim Greedy--Algorithmus am Anfang verusuchen wir, ein Rechteck
$r_i$ im Streifen $g$ zu legen nur, wenn $g = b_i$
(und wenn es eine Lücke zwischen $x_{r_i}$ und $x_{r_i} + s_i$ gibt).
Allerdings müssen wir die Streifen $b_i+1, b_i+2, ..., e_i-1$ vor dem Platzieren prüfen,
ob es in ihnen durchgehend Lücken zwischen $x_{r_i}$ und $x_{r_i} + s_i$ gibt. 
Nur, wenn in allen Streifen $b_i, b_i + 1, ..., e_i-1$ diese Lücken bestehen,
kann das Rechteck $r_i$ in die Platzierung $C'$ eingefügt werden. 
Nachdem alle durch Kollision betroffenen Streifen verarbeitet worden sind,
ist die Platzierung $C'$ fertig.

Dann erfolgt der Vergleich in \cref{line:vergleich}.
Wenn der Gesamtflächeninhalt der Platzierung $C'$ größer, also besser, ist als
der Gesamtflächeninhalt der Platzierung $C$, wird die neue Platzierung vom Algorithmus
akzeptiert und gilt als die aktuelle Platzierung $C$.
Danach muss man offensichtlich alle Lücken und alle nicht gelegten Rechtecke neu bestimmen.
Die Iteratoren $itH$ bzw. $itR$ werden auf $H.begin$ bzw. $U_k.begin$ gesetzt. 
 












