\subsection{Komplexität des Problems}\label{sec:komplexitaet}

%Betrachten wir das zu \fp{} zugehörige Entscheidungsproblem:
%Gegeben ein umschließendes Rechteck $R$ und eine Liste $Z$
%von Rechtecken mit fixierten Länge, Breite und Anordnung entlang der $y$--Achse,
%können alle Rechtecke aus $Z$ innerhalb von $R$ so platziert werden, dass
%sie sich paarweise nicht überdecken?



%Offensichtlich kann dieses Problem von einer nichtdeterministischen
%Turingmaschine bezüglich der Eingabelänge in Polynomialzeit gelöst werden.
%Gegeben sei eine Platzierung der Rechtecke aus $Z$ innerhalb von $R$. 
%Man kann leicht einen in Polynomialzeit laufenden Algorithmus entwickeln, der
%anhand der Koordinaten der kleineren Rechtecke überprüft, ob keines
%der Rechtecke über die Grenzen von $R$ hinausreicht und ob
%kein Paar von Rechtecken aus $Z$ sich überdeckt.  
%Somit liegt \fp{} in NP.

In diesem Abschnitt zeigen wir, dass \fp{} ein Optimierungsproblem ist, das NP--schwer ist.
Es ist unmöglich, einen Algorithmus zu bilden, der in Polynomialzeit prüfen würde,
ob ein Ergebnis zum \fp{} optimal ist. 
Um das zu beweisen, zeigen wir, dass das
0/1--Rucksackproblem zum \textsc{Floh\-markt-\-Pro\-blem} reduziert werden kann.
Das bedeutet: Falls das \textsc{Floh\-markt-\-Pro\-blem} in Polynomialzeit gelöst werden kann,
so auch das Rucksackproblem.
Eine Instanz des Rucksackproblems besteht aus einer Liste von Zahlen
sowie aus einem Rucksack mit einer fixierten Größe.
Das Problem besteht darin, die Zahlen in den Rucksack 
so zu packen, dass ihre Summe maximal ist und sie die Größe des Rucksacks nicht überschreitet.
Das Optimierungsproblem zum 0/1--Rucksackproblem ist NP--schwer.\cite{garey_johnson_2009}


Gegeben sei eine Instanz des Rucksackproblems.
Wir können daraus eine entsprechende Instanz des \textsc{Floh\-markt-\-Pro\-blem}s auf folgende Weise generieren.
Für jede Zahl im Rucksackproblem bilden wir ein Rechteck der Breite 1 und der Länge, die 
dieser Zahl entspricht.
So bilden wir eine Anmeldung im \fp{}, deren Länge der
Zahl aus dem Rucksackproblem entspricht und der Unterschied zwischen
dem Beginn und dem Ende der Anmeldung beträgt eine Stunde.
Außerdem bilden wir ein großes, umschließendes Rechteck, dessen Länge der Größe des Rucksacks entspricht
und dessen Breite ebenfalls 1 beträgt.
Somit bilden wir eine Instanz eines Flohmarkts, der eine Stunde dauert und dessen 
Länge der Größe des Rucksacks entspricht.
In dem hierdurch entstandenen Problem wählen wir die Rechtecke so, 
dass sie über die Grenzen des umschließenden Rechtecks nicht hinausreichen
und der Gesamtflächeninhalt der kleineren Rechtecke maximal ist. 
Insbesondere beachte man, dass man die kleineren Rechtecke nur entlang der längeren
Seite des großen Rechtecks bewegen darf.
Somit ist dieses spezielle \fp{} äquivalent zum ursprünglichen 0/1--Rucksackproblem.
Wenn wir jede Instanz des \fp s in Polynomialzeit lösen könnten,
könnten wir auch jedes 0/1--Rucksackproblem in Polynomialzeit lösen.
Somit ist das \fp{} NP--schwer.

Korf beschreibt ein sehr ähnliches Problem wie das \fp.\cite{korf} 
In seinem Artikel schildert der Autor das \textit{rectangle packing problem}.
In diesem Problem gibt es eine Menge von kleineren Rechtecken und man soll
ein umschließendes Rechteck mit der minimalen Fläche für alle Rechtecke finden,
sodass die Rechtecke sich nicht überdecken.
In diesem Problem dürfen die Rechtecke nicht gedreht werden.
Der Autor beweist, dass das \textit{rectangle packing problem} NP--vollständig ist.

Da das \fp{} NP--schwer ist, muss man über einen optimalen Algorithmus nachdenken. 
Obwohl unser Problem NP--schwer ist, kann es einen parametrisierten Algorithmus geben,
der dieses Problem in Pseudopolynomialzeit lösen würde.
Das betrifft beispielsweise das Rucksackproblem, das sich durch einen dynamischen 
Ansatz in Pseudopolynomialzeit lösen lässt.\cite{parametrized}

Allerdings nutzen wir die Hinweise, die Cormen et al. zum Lösen von NP--vollständigen Problemen
in ihrem Buch bechreiben. Grundsätzlich gibt es drei Ansätze zum Lösen eines
solchens Problems.\cite[S.~1106]{cormen}
Erstens, wenn die Eingabe klein genug ist, reicht ein 
Algorithmus mit einer exponentieller Laufzeit aus.
Allerdings lässt sich diese Idee schlecht umsetzen,
sobald die Anzahl der kleineren Rechtecke in der Eingabe sich in der Ordnung von Hunderten befindet.
Die praktische Laufzeit eines exponentiellen Algorithmus wäre in diesem Fall zu groß.
Zweitens beschreiben die Autoren,
dass man bestimmte Grenzfälle ausgliedern kann, die sich in Polynomialzeit lösen lassen.
Diesen Ansatz verwenden wir bei einigen Beispielen und er wird im \cref{sec:diskussion-ergebnisse}
besprochen.
Drittens kann man einen Algorithmus liefern, der nahezu optimale Ergebnisse 
in Polynomialzeit liefert --- eine Heuristik. 


%\TODO{Zeige, das Problem ist NP (überprüfbar in P)\\
%Zeige, das Problem ist NP-schwer: Reduktion zu einem anderen NP-voll. oder NP-schweren Problem.
%Die Reduktionsfunktion muss in Polynomialzeit laufen.\\
%https://stackoverflow.com/questions/4294270/how-to-prove-that-a-problem-is-np-complete}

%\TODO{Notwendigkeit einer Heuristik}