\subsection{Komplexität des Problems}\label{sec:komplexitaet}

%Betrachten wir das zu \fp{} zugehörige Entscheidungsproblem:
%Gegeben ein umschließendes Rechteck $R$ und eine Liste $Z$
%von Rechtecken mit fixierten Länge, Breite und Anordnung entlang der $y$--Achse,
%können alle Rechtecke aus $Z$ innerhalb von $R$ so platziert werden, dass
%sie sich paarweise nicht überdecken?

Wir zeigen, dass \fp{} NP--vollständig ist, indem wir zunächst zeigen, dass
es in NP liegt und auch NP--schwer ist.

Offensichtlich kann dieses Problem von einer nichtdeterministischen
Turingmaschine bezüglich der Eingabelänge in Polynomialzeit gelöst werden.
Gegeben sei eine Platzierung der Rechtecke aus $Z$ innerhalb von $R$. 
Man kann leicht einen in Polynomialzeit laufenden Algorithmus entwickeln, der
anhand der Koordinaten der kleineren Rechtecke überprüft, ob keines
der Rechtecke über die Grenzen von $R$ hinausreicht und ob
kein Paar von Rechtecken aus $Z$ sich überdeckt.  
Somit liegt \fp{} in NP.

Um zu beweisen, dass das \fp{} NP--schwer ist, zeigen wir, dass das\break
0/1--Rucksackproblem zum \fp{} reduziert werden kann.
Das bedeutet: Falls das\break \fp{} in Polynomialzeit gelöst werden kann,
so kann auch das Rucksackproblem.
Eine Instanz des Rucksackproblems besteht aus einer Liste von Zahlen,
sowie aus einem Rucksack mit einer fixierten Größe.
Das Problem besteht darin, die Zahlen in den Rucksack 
so zu packen, dass ihre Summe maximal ist und sie die Größe des Rucksacks nicht überschreitet.
Das 0/1--Rucksackproblem ist NP--vollständig.\cite{knapsack}
\TODO{sprawdzić/zmienić źródło}


Gegeben sei eine Instanz des Rucksackproblems.
Wir können eine entsprechende Instanz des\break \fp s auf folgende Weise generieren.
Für jede Zahl im Rucksackproblem bilden wir ein Rechteck der Breite 1 und der Länge, die 
dieser Zahl entspricht.
So bilden wir eine Anmeldung im \fp{}, deren Länge der
Zahl aus dem Rucksackproblem entspricht und der Unterschied zwischen
dem Beginn und dem Ende der Anmeldung beträgt eine Stunde.
Außerdem bilden wir ein großes, umschließendes Rechteck, deren Länge der Größe des Rucksacks entspricht
und deren Breite ebenfalls 1 beträgt.
Somit bilden wir eine Instanz eines Flohmarkts, der eine Stunde dauert und deren 
Länge der Größe des Rucksacks enspricht.
In dem hierdurch entstandenen Problem wählen wir die Rechtecke so, 
dass sie über die Grenzen des umschließenden Rechtecks nicht hinausreichen
und der Gesamtflächeninhalt der kleineren Rechtecke maximal ist. 
Insbesondere beachte man, dass man die kleineren Rechtecke nur entalng der längeren
Seite des großen Rechtecks bewegen darf.
Somit ist das \fp{} äquivalent zum ursprünglichen 0/1--Rucksackproblem
Wenn wir jede Instanz des \fp s in Polynomialzeit lösen können,
können wir auch jedes 0/1--Rucksackproblem in Polynomialzeit lösen.
Somit ist das \fp{} NP--schwer und, da es auch in NP liegt, 
ist somit auch NP--vollständig.


Da dieses Problem NP--vollständig ist, muss man über einen optimalen Algorithmus zum
\fp{} nachdenken. 
Cormen et al. bechreiben, dass es grundsätzlich drei Ansätze zum Lösen eines
NP--vollständigen gibt.\cite[S.~1106]{cormen}
Erstens, wenn die Eingabe klein genug ist, reicht ein 
Algorithmus mit einer exponentieller Laufzeit aus.
Allerdings lässt sich diese Idee schlecht umsetzen,
wenn die Anzahl der kleineren Rechtecke in der Eingabe sich in Ordnung von Hunderten befindet.
Die praktische Laufzeit eines exponentiellen Algorithmus ist wäre in diesem Fall zu groß.
Zweitens beschreiben die Autoren,
dass man bestimmte Grenzfälle ausgliedern kann, die sich in Polynomialzeit lösen lassen.
Diesen Ansatz verwenden wir bei einigen Beispielen und er wird im \cref{sec:diskussion-ergebnisse}
besprochen.
Drittens kann man einen Algorithmus liefern, der nahezu optimale Ergebnisse 
in Polynomialzeit liefert --- eine Heuristik.


%\TODO{Zeige, das Problem ist NP (überprüfbar in P)\\
%Zeige, das Problem ist NP-schwer: Reduktion zu einem anderen NP-voll. oder NP-schweren Problem.
%Die Reduktionsfunktion muss in Polynomialzeit laufen.\\
%https://stackoverflow.com/questions/4294270/how-to-prove-that-a-problem-is-np-complete}

%\TODO{Notwendigkeit einer Heuristik}