\section{Beispiele}
Das Format der Eingabe wurde geändert. 
Die erste Zeile beinhaltet die Zahl $N$.
Die zweite Zeile enthält eine Folge von Öffnungszeiten des Flohmarkts.
Die Uhrzeit mit einem geradem Index ist jeweils der Beginn und die Uhrzeit 
mit einem ungeraden Index ist jeweils das Ende.\footnote{Indexierung beginnt mit 0.}
Die dritte Zeile enthält die Anzahl der Anmeldungen $n$ und es folgen $n$ Zeilen mit den Anmeldungen.

Um das Zeitformat zu ändern, müssen die Zahlen im Format \ttt{H:MM} angegeben werden, wobei 
die \ttt{H} eine beliebige natürliche Zahl (mit 0) sein kann und
die Minuten \ttt{MM} immer als eine zweistellige Zahl dargestellt werden müssen.

Der Übersichtlichkeit halber befinden sich die genauen Standorte zu Anmeldungen
zu jedem Beispiel in den angehängten csv--Dateien.
In den Tabellen entsprechen die Werte $x_1$ und $x_2$ den Werten 
$x_r$ und $x_r + s_r$ jedes Rechtecks $r$.

\subsection{Beispiel 1}\label{ex:1}
Der Flächeninhalt des großen Rechtecks: \framebox[1.1\width]{10000 [m $\cdot$ h]}\\
Der Gesamtflächeninhalt aller platzierten Rechtecke: \framebox[1.1\width]{8028 [m $\cdot$ h]}\\
Der Gesamtflächeninhalt aller Rechtecke: \framebox[1.1\width]{8028 [m $\cdot$ h]}
\subsection{Beispiel 2}\label{ex:2}
Der Flächeninhalt des großen Rechtecks: \framebox[1.1\width]{10000 [m $\cdot$ h]}\\
Der Gesamtflächeninhalt aller platzierten Rechtecke: \framebox[1.1\width]{9077 [m $\cdot$ h]}\\
Der Gesamtflächeninhalt aller Rechtecke: \framebox[1.1\width]{10002 [m $\cdot$ h]}
\subsection{Beispiel 3}\label{ex:3}
Der Flächeninhalt des großen Rechtecks: \framebox[1.1\width]{10000 [m $\cdot$ h]}\\
Der Gesamtflächeninhalt aller platzierten Rechtecke: \framebox[1.1\width]{8778 [m $\cdot$ h]}\\
Der Gesamtflächeninhalt aller Rechtecke: \framebox[1.1\width]{10010 [m $\cdot$ h]}
\subsection{Beispiel 4}\label{ex:4}
Der Flächeninhalt des großen Rechtecks: \framebox[1.1\width]{10000 [m $\cdot$ h]}\\
Der Gesamtflächeninhalt aller platzierten Rechtecke: \framebox[1.1\width]{7370 [m $\cdot$ h]}\\
Der Gesamtflächeninhalt aller Rechtecke: \framebox[1.1\width]{10534 [m $\cdot$ h]}
\subsection{Beispiel 5}\label{ex:5}
Der Flächeninhalt des großen Rechtecks: \framebox[1.1\width]{10000 [m $\cdot$ h]}\\
Der Gesamtflächeninhalt aller platzierten Rechtecke: \framebox[1.1\width]{8705 [m $\cdot$ h]}\\
Der Gesamtflächeninhalt aller Rechtecke: \framebox[1.1\width]{30940 [m $\cdot$ h]}
\subsection{Beispiel 6}\label{ex:6}
Der Flächeninhalt des großen Rechtecks: \framebox[1.1\width]{10000 [m $\cdot$ h]}\\
Der Gesamtflächeninhalt aller platzierten Rechtecke: \framebox[1.1\width]{10000 [m $\cdot$ h]}\\
Der Gesamtflächeninhalt aller Rechtecke: \framebox[1.1\width]{10000 [m $\cdot$ h]}
\subsection{Beispiel 7}\label{ex:7}
Der Flächeninhalt des großen Rechtecks: \framebox[1.1\width]{10000 [m $\cdot$ h]}\\
Der Gesamtflächeninhalt aller platzierten Rechtecke: \framebox[1.1\width]{9979 [m $\cdot$ h]}\\
Der Gesamtflächeninhalt aller Rechtecke: \framebox[1.1\width]{10000 [m $\cdot$ h]}
\subsection{Beispiel 8}\label{ex:8}
Besonderheit: Die Länge $N$ und der Zeitraum von $B$ bis $E$ des Flohmarkts sind
verschieden von der Aufgabenstellung. 
\vspace{.4cm}

\noindent Der Flächeninhalt des großen Rechtecks: \framebox[1.1\width]{4776 [m $\cdot$ h]}\\
Der Gesamtflächeninhalt aller platzierten Rechtecke: \framebox[1.1\width]{4427 [m $\cdot$ h]}\\
Der Gesamtflächeninhalt aller Rechtecke: \framebox[1.1\width]{17228 [m $\cdot$ h]}
\subsection{Beispiel 9}\label{ex:9}
Besonderheit: Der Flohmarkt dauert zwei Tage: Jeweils von 8:00 bis 18:00.
Der Zeitraum des Flohmarkts ist dementsprechend unterborchen.
\vspace{.4cm}

\noindent Der Flächeninhalt des großen Rechtecks: \framebox[1.1\width]{7600 [m $\cdot$ h]}\\
Der Gesamtflächeninhalt aller platzierten Rechtecke: \framebox[1.1\width]{7594 [m $\cdot$ h]}\\
Der Gesamtflächeninhalt aller Rechtecke: \framebox[1.1\width]{10000 [m $\cdot$ h]}
\subsection{Beispiel 10}\label{ex:10}
Besonderheit: Das Zeitformat ist in Minuten.
\vspace{.4cm}

\noindent Der Flächeninhalt des großen Rechtecks: \framebox[1.1\width]{95274 [m $\cdot$ min]}\\
Der Gesamtflächeninhalt aller platzierten Rechtecke: \framebox[1.1\width]{61796 [m $\cdot$ min]}\\
Der Gesamtflächeninhalt aller Rechtecke: \framebox[1.1\width]{68611 [m $\cdot$ min]}

\todo[inline,caption={Beispiele}]{
  Weitere Beispiele:
  \begin{itemize}
    \item s. Abschnitt Konversion
    \item andere Zeiten \checkmark (Bsp 8)
    \item andere Länge \checkmark (Bsp 8)
    \item mehrtägiger Flohmarkt + unterbrochener Zeitraum \checkmark (Bsp 9)
    \item Minuten \checkmark (Bsp 10)
    \item edge--cases, bei denen der Algorithmus nicht funktioniert
  \end{itemize}
}
