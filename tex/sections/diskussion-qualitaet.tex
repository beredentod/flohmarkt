\subsubsection{Qualität der Ergebnisse}\label{sec:diskussion-ergebnisse}
Im vorangegangenen Abschnitt werden die Grenzen der Heuristik erkannt. 
In diese Abschnitt diskutieren wir die Qualität der herausgekommenen Ergebnisse.

Zuerst bestimmen wir die Kriterien, unter denen wir ein Ergebnis auswerten:
\begin{itemize}
	\item Der Verhältnis vom Gesamflächeninhalt der ins große Rechteck gelegten Rechtecke 
	zu Gesamflächeninhalt aller Rechtecke.
	\item Da Verhältnis vom Gesamflächeninhalt der ins große Rechteck gelegten Rechtecke 
	zum Flächeninhalt des großen Rechtecks.
	\item Die praktische Laufzeit des Programms für ein Ergebnis.
\end{itemize}

Diese Kriterien wurden in Bezug auf die Aufgabenstellung gewählt,
„um den Veranstaltern des Flohmarkts zu helfen“.
Das erste Kriterium gibt den Veranstaltern den Einblick darin,
wie viel sie in Bezug auf die Menge des verfügbaren Geldes verdienen ---
alle Personen wollen den Veranstaltern Geld anbieten, aber es hängt von den 
Organisatoren ab, welche Anmeldungen sie ablehnen und welche annehmen.
Das zweite Kriterium gibt den Einblick darin, wie viel Geld die 
Veranstalter verdienen in Bezug auf den verfügbaren Platz. 
Die beiden ersten Kriterien liefern natürlich auch die Erkenntnis über die Verlüste, die 
mit der Auswahl an Anmeldungen verbunden sind.

Das dritte Kriterium spielt für die Veranstalter eine praktische Rolle.
Sehr wenigen Personen würde ein Ergebnis interessieren, das vielleicht um ein paar Prozent 
besser ist, aber dessen Bestimmung mehrere Stunden (oder Tage!) dauert. 
Deshalb wurde auch die Brute--Force--Lösung ausgeschlossen.

\begin{table}[h]
\centering
\begin{tabular}{|>{\ttfamily}l|>{\ttfamily}l|c|c|c|c|c|c|c|}
\hline
\textnormal{Kriterien I} & \textnormal{Kriterien II} & Bsp. 1 & Bsp. 2 & Bsp. 3 & Bsp. 4 & Bsp. 5 & Bsp. 6 & Bsp. 7 \\
\hline
greaterHolesSize & smallerSize & 8028 & 9077 & 8778 & 7370 & 8705 & 10000 & 9979 \\
\hline
greaterHolesSize & greaterSize & 8028 & 9077 & 8778 & 7370 & 8705 & 10000 & 9973 \\
\hline
greaterHolesSize & smallerArea & 8028 & 9077 & 8778 & 7370 & 8705 & 10000 & 9979 \\
\hline
greaterHoleaSize & greaterSize & 8028 & 9077 & 8778 & 7370 & 8705 & 10000 & 9973 \\
\hline
greaterHoleaSize & greaterEnd & 8028 & 9077 & 8778 & 7370 & 8705 & 10000 & 9973 \\
\hline
smallerHolesSize & smallerSize & 8028 & 9077 & 8778 & 7370 & 8599 & 10000 & 9980 \\
\hline
smallerHolesSize & greaterSize & 8028 & 9077 & 8778 & 7370 & 8599 & 10000 & 9973 \\
\hline
smallerHolesSize & smallerArea & 8028 & 9077 & 8778 & 7370 & 8599 & 10000 & 9980 \\
\hline
smallerHolesSize & greaterArea & 8028 & 9077 & 8778 & 7370 & 8599 & 10000 & 9973 \\
\hline
smallerHolesSize & greaterEnd & 8028 & 9077 & 8778 & 7370 & 8599 & 10000 & 9991 \\
\hline
\end{tabular}
\end{table}

\TODO{Abglühen\\
Ergebnisse einzeln beschreiben\\
Ergebnisse nach dem Greedy-Algo\\
welches Ergebnis konnte (nicht) verbessert werden?}