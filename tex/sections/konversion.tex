\subsubsection{Konversion der Eingabe}
Wie schon im \cref{sec:definitionen} erwähnt wurde, können die Gedanken bezüglich des
\fp{} auf andere Größen übetragen werden. Da die Größen des Rechtecks $R$ sowie des Zeitraums fest sind
und auf 1000 Metern bzw. auf den Zeitraum von 8:00 is 18:00 beschränkt sind, konvertieren wir
die Eingabe, indem wir den Beginn des Zeitraumes auf 0 setzen und wir den 
Beginn $B$ vom Ende $E$ subtrahieren.
So bleibt auch der Wert $M$, also die Differenz von $E$ und $B$, gleich. 
Analog müssen wir die Eingabe für die kleineren Rechtecke $r_i$ entsprechend konvertieren, indem wir
von jedem $b_i$ und $e_i$ den Wert $B$ abziehen. Für die Aufgabe selbst hat diese Konversion keine
Bedeutung und funktioniert auch, wenn ein angegebener Zeitraum sich vom ursprünglichen Zeitraum unterscheidet.

Mit dieser Konversion können wir ebenfalls mehrtägige Flohmärkte oder
sogar mehrere unterschiedlichen Flohmärkte behandeln.
Zur Darstellung eines mehrtägigen Flohmarktes kann man die gesamte Öffnungszeiten des Flohmarktes in Stunden angeben,
z.B. der Zeitraum eines Flohmarkts, der zwei Tage von 10:00 bis 17:00 dauert,
kann als von 10:00 bis 41:00 (17:00 + 24 Stunden) dargestellt werden. 
Dann ist der Zeitraum von 17:00 bis 34:00 an keiner Stelle besetzt.
Ebenfalls, wenn ein angegebener Zeitraum an einer Stelle unterbrochen ist,
etwa dauert der Flohmarkt von 7:00 bis 9:00 und dann von 12:00 bis 15:00, kann der
Zeitraum von 7:00 bis 15:00 angegeben werden und alle Anmeldungen, die zumindest zum Teil
in der Pausenzeit liegen, können aus der Eingabe entfernt werden oder können gar nicht
angegeben werden.
Mehrere unterschiedlichen Flohmärkte kann man analog kodieren. Es hängt nur von der Eingabe ab.


Außerdem werden im ursprünglichen Problem alle Zeiten in vollständigen Stunden angegeben.
Diese Aufgabe kann sehr leicht zu Zeiten in Minuten ergänzt werden, indem man die Eingabe
am Anfang entsprechnd konvertiert: Man kann einfach alle Zeiten zu Minuten umrechnen, indem
man vollständige Stunden mal 60 multipliziert.
Obwohl die weiteren Betrachtungen sich grundsätzlich auf vollständige Stunden beziehen
(wie in der Aufgabenstellung),
soll man nicht vergessen, dass alle diesen Gedanken sich auf Minuten übertragen lassen.