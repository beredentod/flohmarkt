\documentclass[a4paper,10pt,ngerman]{scrartcl}
\usepackage{babel}
\usepackage[T1]{fontenc}
\usepackage[utf8]{inputenc}
\usepackage{textcomp}
\usepackage[a4paper,margin=2.5cm,footskip=0.5cm]{geometry}

% Die nächsten drei Felder bitte anpassen:
\newcommand{\Aufgabe}{Aufgabe 1: Flohmarkt} % Aufgabennummer und Aufgabennamen angeben
\newcommand{\TeilnahmeId}{55628}       % Teilnahme-Id angeben
\newcommand{\Namen}{Michal Boron} % Namen der Bearbeiter/-innen dieser Aufgabe angeben
 
% Kopf- und Fußzeilen
\usepackage{scrlayer-scrpage, lastpage}
\setkomafont{pageheadfoot}{\large\textrm}
\lohead{\Aufgabe}
\rohead{Teilnahme-Id: \TeilnahmeId}
\cfoot*{\thepage{}/\pageref{LastPage}}

% Position des Titels
\usepackage{titling}
\setlength{\droptitle}{-1.0cm}
\usepackage{seqsplit}
\usepackage{verbatim}

% Für mathematische Befehle und Symbole
\usepackage{amsmath}
\usepackage{amssymb}
\usepackage{amsthm}
%\usepackage{cite}

\usepackage[backend=bibtex]{biblatex}
\addbibresource{flohmarkt.bib}

\usepackage{hyperref}
\hypersetup{
    colorlinks=false,
    linkcolor=blue,
    filecolor=magenta,      
    urlcolor=cyan,
}
% Für Bilder
\usepackage{graphicx}
\graphicspath{ {./tex/images/} }

% Für Algorithmen
\usepackage{algpseudocode}
\usepackage{algorithm}
\usepackage{gensymb}
\usepackage{tikz}
\usepackage{caption}
\usepackage{subcaption}
\usepackage{array}
\usepackage{makecell}

\usepackage[backgroundcolor=lightgray]{todonotes}
\usepackage{minibox}


\usepackage{enumitem}
\usepackage[export]{adjustbox}
\usepackage{csquotes}

\usepackage[backend=bibtex]{biblatex}
\addbibresource{spiess.bib}

% Für Quelltext
\usepackage{listings}
\usepackage{color, colortbl}
\definecolor{mygreen}{rgb}{0,0.6,0}
\definecolor{mygray}{rgb}{0.5,0.5,0.5}
\definecolor{mymauve}{rgb}{0.58,0,0.82}
\definecolor{lightblue}{HTML}{cce6ff}
\definecolor{lightred}{HTML}{ffb3b3}

\lstset{
  keywordstyle=\color{blue},commentstyle=\color{mygreen},
  stringstyle=\color{mymauve},rulecolor=\color{black},
  basicstyle=\footnotesize\ttfamily,numberstyle=\tiny\color{mygray},
  captionpos=b, % sets the caption-position to bottom
  keepspaces=true, % keeps spaces in text
  numbers=left, numbersep=5pt, showspaces=false,showstringspaces=true,
  showtabs=false, stepnumber=2, tabsize=2
}

% Diese beiden Pakete müssen zuletzt geladen werden
\usepackage{hyperref} % Anklickbare Links im Dokument
\usepackage{cleveref}

\newtheorem{lemma}{Lemma}
\newtheorem{definition}{Definition}
\newtheorem{satz}{Satz}
\newtheorem{axiom}{Axiom}
\newtheorem{korollar}{Korollar}

\newcommand{\TODO}[1]{\todo[inline]{TODO: #1}}
\newcommand{\mb}[1]{{\color{red}[MB: #1]}}
\newcommand{\tbf}[1]{\textbf{#1}}
\newcommand{\ttt}[1]{\texttt{#1}}

\usetikzlibrary{fit,backgrounds,positioning}
\tikzset{vertex/.style={circle,draw,minimum size=0.8cm,inner sep=1pt,fill=white}}

% Daten für die Titelseite
\title{\textbf{\Huge\Aufgabe}}
\author{\LARGE Teilnahme-Id: \LARGE \TeilnahmeId \\\\
	    \LARGE Bearbeiter dieser Aufgabe: \\ 
	    \LARGE \Namen\\\\}
\date{\LARGE April 2021}

\begin{document}

\maketitle
\tableofcontents

\todo[inline,caption={Struktur}]{
  \begin{itemize}
    \item Definitionen, Modellierung des Problems
    \item (Themenbezogene Arbeiten)
    \item Komplexität
    \begin{itemize}
      \item Notwendigkeit einer Heuristik
    \end{itemize}
    \item heuristisches Verfahren
    \begin{itemize}
      \item Greedy--Anlegen am Anfang
      \item heurostisches Verbesserungsverfahren
      \begin{itemize}
        \item welche Methode? 
        \item hill climbing
        \item simuliertes Abglühen
      \end{itemize}      
    \end{itemize}
    \item Diskussion der Ergebnisse
    \begin{itemize}
      \item Grenzen/Mängel der Heuristik
      \begin{itemize}
        \item was wird nicht erkannt? (edge--cases)
        \item was lässt sich nicht eindeutig ausschließen?
        \item getroffene Annahmen
      \end{itemize}
      \item Qualität der Ergebnisse
      \begin{itemize}
        \item Qualität der Ergebnisse am Anfang (Greedy--Verfahren)
        \item Qualität bzgl. des großen Flächeninhalt, des Gesamtflächeninhalts aller Rechtecke, \%
        \item was und wann kann nicht verbessert werden? (Beispiel 4: 7370)
      \end{itemize}      
    \end{itemize}
    \item Laufzeit
  \end{itemize}
}

\section{Lösungsidee}

\subsection{Formulierung des Problems}
Gegeben sei eine Strecke der Länge $N$ und eine Zeitspanne von $B$ bis $E$.
Außerdem gegeben sei eine Liste von $Z$ Voranmeldungen. 
Die Voranmeldugen betreffen die Vermietung eines Teils der Strecke in einer konkreten Zeitspanne.
So besteht jede Voranmeldug $i$ aus einer Strecke $0 < s_i \leqslant N$,
einem Mietbeginn $B \leqslant b_i < E$ und einem Mietende $b_i < e_i \leqslant E$.
In diesem Problem behandelt werden Strecken in volltändigen Metern 
und alle Zeiten werden in vollständigen Stunden angegeben.
Obwohl $N$ auf 1000 Meter, $B$ auf 8:00 und $E$ auf 18:00 festgelegt sind,
kann mein Programm mit beliebigen Größen umgehen.

Die Aufgabe ist ein Optimierungsproblem.
Man soll so eine Teilfolge aus den $m$ Voranmeldugen wählen,
dass alle gewählten Strecken in den angebenen Zeiten vermietet werden können
und die Mieteinnahmen möglichst hoch sind, wobei der Preis 1 Euro pro Meter pro Stunde beträgt.

Man kann das Problem auf folgende Weise modellieren. 
Wir setzen: $M := E - B$.
Wir bilden ein Rechteck $R$ der Größe $N \times M$.
So kann man analog jede Voranmeldung $i$ als ein kleineres Rechteck 
$r_i$ der Größe $s_i \times m_i$ darstellen, wobei $m_i := e_i - b_i$.

So können wir die obige Aufgabe umformulieren:
Wähle so eine Teilfolge $Z'$ von Rechtecken aus $Z$,
die eine Anordnung innerhalb von $R$ bilden,
sodass der Gesamtflächeninhalt aller Rechtecke in $Z'$ maximal ist und
kein Paar der Rechtecke sich nicht überdeckt.
Genauer gesagt: Jedes Rechteck $r_i$ in $Z'$ besitzt Ecken,
die den folgenden Punkten entsprechen:
$(x_i, b_i), (x_i, e_i), (x_i + s_i, e_i), (x_i + s_i, b_i)$.

\TODO{check, reformulate}

\subsection{Themenbezogene Arbeiten}

\subsection{Komplexität des Problems}

\TODO{Zeige, das Problem ist NP (überprüfbar in P)\\
Zeige, das Problem ist NP-schwer: Reduktion zu einem anderen NP-voll. oder NP-schweren Problem.
Die Reduktionsfunktion muss in Polynomialzeit laufen.\\
https://stackoverflow.com/questions/4294270/how-to-prove-that-a-problem-is-np-complete}


\TODO{Notwendigkeit einer Heuristik}

\subsection{Inhalt: Heuristik}

\subsection{Grenzen der Heuristik}

\subsection{Laufzeit}




\section{Umsetzung}\label{sec:umsetzung}

\section{Beispiele}

\section{Quellcode}
\lstinputlisting[language=C++]{./tex/flohmarkt.m}

\end{document}