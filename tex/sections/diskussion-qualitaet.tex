\subsubsection{Qualität der Ergebnisse}\label{sec:diskussion-ergebnisse}
Im vorangegangenen Abschnitt werden die Grenzen der Heuristik erkannt. 
In diesem Abschnitt diskutieren wir die Qualität der erhaltenen Ergebnisse.

Zuerst bestimmen wir die Kriterien, unter denen wir ein Ergebnis auswerten:
\begin{itemize}
	\item Das Verhältnis vom Gesamtflächeninhalt der ins große Rechteck gelegten Rechtecke 
	zum Gesamtflächeninhalt aller Rechtecke.
	\item Das Verhältnis vom Gesamtflächeninhalt der ins große Rechteck gelegten Rechtecke 
	zum Flächeninhalt des großen Rechtecks.
	\item Die praktische Laufzeit des Programms für ein Ergebnis.
\end{itemize}

Diese Kriterien wurden in Bezug auf die Aufgabenstellung gewählt,
„um den Organisatoren des Flohmarkts zu helfen.“
Das erste Kriterium gibt den Veranstaltern den Einblick darin,
wie viel sie in Bezug auf die Menge des verfügbaren Geldes verdienen ---
alle Mieter wollen den Veranstaltern Geld anbieten, aber es hängt von den 
Organisatoren ab, welche Anmeldungen sie ablehnen und welche annehmen.
Das zweite Kriterium gibt den Einblick darin, wie viel Geld die 
Veranstalter verdienen in Bezug auf den verfügbaren Platz. 
Die beiden ersten Kriterien liefern natürlich auch die Erkenntnis über die Verluste, die 
mit der Auswahl an Anmeldungen verbunden sind.
Das dritte Kriterium spielt für die Veranstalter eine praktische Rolle.
Sehr wenige Personen würde ein Ergebnis interessieren, das vielleicht um ein paar Promile 
besser ist, aber dessen Bestimmung mehrere Stunden (oder Tage!) dauert. 
Deshalb wurde auch die Brute--Force--Lösung ausgeschlossen.

\cref{tab:ergebnisse} stellt die Ergebnisse zu den Beispielen von \hyperref[ex:1]{1} bis \hyperref[ex:7]{7}
aus der BWINF--Webseite dar.
In den ersten zwei Spalten befinden sich entsprechend die Sortierkriterien für Liste
$H$ und die Listen $U_j$. Die Sortierkriterien werden in der \cref{tab:kriterien} erläutert.
Die erste Zeile in der Tabelle stellt die Ergebnisse bevor dem Lauf des Verbesserungsverfahrens dar.
Die Zahlen zwischen den Spalten „Bsp. 1“ und „Bsp. 7“ sind die Ergebnisse in [m $\times$ h]
(nach der Aufgabenstellung: auch in Euro),
die das Programm unter Verwendung von den angegebenen Sortierkriterien liefert.
Dazu ist zu beachten, dass der Flächeninhalt des Rechtecks $R$ in allen abgebildeten Beispielen
10000 beträgt.
 
\input{./tex/other/tab_ergebnisse}
\input{./tex/other/tab_kriterien}

Man kann leicht feststellen, dass der Greedy--Algorithmus sehr gute Ergebnisse liefert, die 
manchmal nur um winzige Prozentpunkte durch den Verbesserungsalgorithmus verbessert werden.

Im Beispiel 1 werden alle Rechtecke bereits beim Lauf des Greedy--Algorihtmus am Anfang platziert.
Wenn man den Gesamtflächeninhalt aller Rechtecke aus diesem
Beispiel ausrechnet, kommt man auf die Zahl 8028.
Das ist also das optimale Ergebnis.
Ebenfalls ist beim Beispiel 6 sehr leicht zu erkennen, dass 
der Gredy-Algorithmus das optimale Ergebnis liefert, da alle 
Rechtecke aus dem Beispiel platziert und die Fläche des großen Rechtecks vollständig bedeckt wurden.

Das Beispiel 3 ist ein sehr interessanter Fall. 
Weder werden alle Rechtecke aus diesem Beispiel platziert (das ist sowieso unmöglich, da der Gesamtflächeninhalt aller Rechtecke 10010 beträgt), noch ist 
die Fläche des großen Rechtecks völlig bedeckt. 
Allerdings, wenn wir die Ergebnisse jedes Streifens einzeln betrachten, stellen wir fest,
dass der Flohmarkt von 11:00 bis 17:00 vollständig durchgehend ausgebucht ist, also beträgt
der Flächeninhalt jedes Streifens von 3 bis 8 (einschließlich) 1000.
Das bedeutet, man kann das Ergebnis für diese Streifen nicht verbessern. 
Man kann auch feststellen, dass alle Rechtecke, die zu den Streifen 0, 1, 2, 9 gehören,
platziert wurden.
Das bedeutet, dass die restliche Fläche
mit keinen Rechtecken bedeckt werden kann. Das bedeutet, dass 8778 das optimale Ergebnis für dieses
Beispiel ist, weil man dieses Ergebnis nicht verbessern kann.

Die Situation mit dem Beispiel 4 sieht ähnlich aus. 
Obwohl nicht alle Rechtecke platziert werden (wieder beträgt der Gesamtflächeninhalt aller
Rechtecke mehr als 10000) und es noch viel freien Platz im großen Rechteck gibt (mehr als ein Viertel des
Flächeninhalts), ist dieses Ergebnis optimal.
Man kann per Hand prüfen, dass jede Kombination mit den zwei nicht gelegten
Rechtecken kein besseres Ergebnis ergibt.

Dann bleiben noch die Ergebnisse zu den Beispielen 2, 5 und 7. 
Da die Anzahl an kleineren Rechtecken zu groß ist, um die Ergebnisse per Hand zu prüfen und
wir keinen Brute--Force--Algorithmus laufen lassen, kann man schwer sagen,
wie weit die Ergebnisse von den Optima abweichen. 
Zur Auswertung dieser Ergebnisse verwenden wir unsere festgelegten Kriterien.

Das Ergebnis zum Beispiel 2 vor dem Lauf des Greedy-Algorithmus am Anfang 
beträgt 9056. Dieser Gesamtflächeninhalt entspricht 90,5\% des Flächeninhalts des
großen Rechtecks und ebenfalls 90,5\% des Gesamtflächeninhalts aller Rechtecke in diesem Beispiel.
Man beachte, dass der Gesamtflächeninhalt aller Rechtecke 10000 überschreitet.
Durch den Verbesserungsalgorithmus steigt das Anfangsergebnis auf 9077,
also eine Verbesserung um 0,2 Prozentpunkte.
In diesem Beispiel gibt es 603 Rechtecke und das sind Anmeldungen, deren Wert $s_i$ hauptsächlich zwischen 
1 und 6 liegt. Die praktische Laufzeit des Algorihtmus liegt im Bereich von 20 Sekunden
auf einem modernen Recher.\footnote{Die genaue Laufzeit ist nicht interessant, ohne dass die 
technischen Spezifikationen des Rechners angegeben werden.}
Der Wert 90,7\% ist akzeptabel in Bezug auf die möglichen Einkommen und auf den verfügbaren Platz
und aus praktischer Sicht ist es schwer, sich ein besseres Ergebnis innerhalb von 20 Sekunden zu wünschen,
solange die Ermittlung eines qualitativen Ergbnisses per Hand sehr lange dauern würde.
Es ist anhand der \cref{tab:ergebnisse} festzustellen, dass das Ergebnis sich nach dem Lauf des
Verbesserungsalgorithmus unter Verwendung von beliebigen Kriterien um denselben Wert verbessert.

Das Anfangsergebnis zum Beispiel 7 ist 9959 und beträgt genau 99,59\% des Flächeninhalts
des Rechtecks $R$ und des Gesamtflächeninhalts aller Rechtecke. In diesem Beispiel
gibt es 566 Anmeldungen, davon haben die meisten den Wert $s_i$ im Bereich von 1 bis 6, aber
im Vergleich zum Beispiel 2 gibt Rechtecke, die einen Wert $s_i$ im Bereich 10--40 besitzen.
Wenn man die Sortierkriterien \texttt{greaterHolesSize} und \texttt{smallerSize} wählt,
also diejenigen, die für das Program gewählt wurden und die im \cref{sec:verbesserung} beschrieben werden,
liefert das Verbesserungsverfahren ein Ergebnis um 0,2 Prozentpunkte besser.
Wir stellen fest, die anderen Kombinationen der Sortierkriterien und insbesondere die Kombination
aus der letzen Zeile der Tabelle ein besseres Ergebnis liefert.
Das bedeutet, dass der Wert 9978 auf jeden Fall nicht optimal ist. 
Allgeimein ist der Wert 9978 aus praktischer Sicht völlig akzeptabel.
Es ist kaum möglich, innerhalb von einer Sekunde so einen Wert zu erreichen,
wenn man eine Platzierung der Rechtecke per Hand bestimmt.
Aus praktischer Sicht ist der Wert 9991 auch nicht von einem bedeutenden Unterschied zu 9978 und
sind die beiden Werte sehr nah am Optimum.

Als letztes bleibt das Beispiel 5, das 25 große Rechtecke umfasst, und zu dem 
der Greedy--Algorithmus am Anfang das Ergebnis 7962 liefert. 
Der Gesamtflächeninhalt aller Rechtecke beträgt mehr als 300000 und somit
entspricht das Ergebnis 79,6\% des Flächeninhalts des Rechtecks $R$ und
25,7\% des Gesamtflächeninhalts aller Rechtecke.
Das Verbesserungsverfahren unter Verwendung von den Sortierkriterien \texttt{greaterHolesSize}
liefert einen Wert 8705, also ist das eine Verbesserung um 7,4\% Prozentpunkte.
Hingegen beträgt das verbesserte Ergebnis nur 8599, wenn man 
die Sortierkriterien \texttt{smalllerHolesSize} verwendet. 
Es ist schwer zu beurteilen, wie weit das Ergebnis vom Optimum abweicht. 
Allerdings ist 87\% des Flächeninhalts des großen Rechtecks, also des verfügbaren Platzes,
noch in Ordnung und so ein Wert ist aus praktischer Sicht wünschenswert. 
Nebenbei bemerkt liegt dieser Wert unter dem Grundfreibetrag, was bedeutet, dass
die Veranstalter eigentlich keine Steuer abführen müssen,
wenn sie dieses Jahr keine weiteren Einnahmen planen. 
Wenn der Wert über 9400 Euro liegen würde, müsste man noch die 
Steuer bezahlen und die Eikommen wären geringer :-).

Die deutlichste Verbesserung im Beispiel 5 liegt an der Entscheidung,
die Sortierkriterien\break \texttt{greaterHolesSize} und \texttt{smallerSize} zu wählen.
Eventuell könnte man den Algorithmus so entwickeln, dass man ihn 10--mal laufen lässt
und in jedem Lauf andere Kriterien wählt. Dann könnte einfach das Maxmimum aus 
diesen Ergebnissen gezogen werden. In vielen Heuristiken ist es genau die Idee, verschiedene Methoden
zusammen zu verknüpfen.
Bei allen Beispielen außer 2 dauert die Laufzeit des ganzen Programm etwa eine Sekunde,
was praktisch eine sehr gute Laufzeit darstellt, und die Qualität der Ergebnisse 
ist in diesem Zusammenhang sehr gut, wenn man dies mit einer faktoriellen Laufzeit 
eines Brute--Force--Lösung vergleicht. Die durch den Bergsteigerlagorithmus 
gefundenen Maxima liegen höchstwahrscheinlich sehr nah an den Optima.

Zum Vergleich wurde auch ein Ansatz mit der simulierten Abkühlung und ein Ansatz, in
dem die Reihenfolgen der Rechtecke in den Listen zufällig war, ausprobiert, aber
die Ergebnisse waren allgemein schlechter als die, die der Quasibergsteigeralgorithmus liefert.