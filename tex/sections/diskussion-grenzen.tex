\subsubsection{Grenzen der Heuristik}\label{sec:diskussion-grenzen}
Im \cref{sec:komplexitaet} wurde bewiesen, dass das \fp{} NP--vollständig ist.
Es gibt $|Z|!$ mögliche Anordnungen der Rechtecke.
Deshalb für $|Z|$ in der Ordnung von ca. 700 wird eine Brute--Force--Lösungin einer nicht
akzeptablen Zeit gelöst.
Das ist auch der Grund, warum wir eine Heurstik verwenden.
Allerdings, da eine Heuristik ein Approxiamationsalgorithmus ist und
nur nahezu optimale Ergebnisse liefert, muss es Kompromisse geben.
Dieser Kompromiss betrifft vor allem die Laufzeit und 
dafür, dass der Algorithmus in Polynomialzeit läuft, 
trifft das Programm an vielen Stellen vereinfachte Entscheidungen,
die nicht zum optimalen Ergebnis führen.
In diesem Abschnitt diskutieren wir nur über die Grenzen der Heuristiken im Programm
und im \cref{sec:diskussion-ergebnisse} besprechen wir die Ergebnisse.


Im Greedy--Algorithmus am Anfang liegt die Schwierigkeit darin, dass
die Platzierung der Rechtecke grundsätzlich von ihrer Reihenfolge 
in Listen $S_j$ abhängt. Diese hängt dann von den Sortierkriterien ab. 
Obwohl dank der gewählten Sortierkriterien optimale oder sehr gute Ergebnisse
bei vielen Beispielen herauskommen, ist das nicht der Fall bei allen Beispielen
(mehr dazu im \cref{sec:diskussion-ergebnisse}).
Auf jeden Fall liefert der Greedy--Algorithmus kein optimales Ergebnis zum \nameref{ex:2},
weil dieses Ergebnis im Laufe des Verbesserungsverfahrens verbessert wird.


Im Verbesserungsverfahren wurden mehrere Kompromisse zugunsten der Laufzeit gemacht.
Vor allem liegt die Schwierigkeit zugrunde dem Verfahren --- warum ein Bergsteigeralgorithmus
und nicht z.B. ein Verfahren mit simuliertem Abglühen oder ein ganz anderer heuristische Ansatz?
Außerdem liegen die Schwierigkeiten des Verbesserungsverfahrens auch an
der Reihenfolgen der Listen $U_j$ und der Liste $H$.
Zum Sortieren dieser Listen nutzt man auch festgelegte Sortierkriterien,
die nicht zwingend das optimale Ergebnis liefern müssen.
Dazu liegt das Problem auch beim Platzieren eines Rechtecks in eine Lücke. 
Wir entscheiden uns, das Rechteck an die Koordinate $x_2$ der Lücke zu legen.
Man könnte hier eine andere Vorgensweise anwenden, z.B. man könnte das
Rechteck an die Koordinate $x_1$ orientieren, man könnte auf eine besondere
Weise vorgehen, wenn das Rechteck deutlich kleiner ist als die Lücke, oder
man könnte die Rechtecke in darüber und darunter stehenden Streifen in Betracht ziehen.
Wie beim Greedy--Algorithmus am Anfang, hängt die Reihenfolge der Rechtecke 
beim Ausfüllen der Lücken von der Reihenfolge in Listen $S_j$ ab. 