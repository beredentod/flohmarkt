\subsection{Klasse \ttt{Solver}}
In dieser Klasse wird das vorgestellte Verfahren implementiert.

Die Attribute \ttt{N} und \ttt{M} entsprechen den Zahlen $N$ und $M$
aus dem \cref{sec:definitionen}. Es gibt das Attribut \ttt{START}, das dem
Wert $B$ entspricht. Die Variable \ttt{recess} entspricht der
Dauer der Schließungszeit des Flohmarkts, wenn sein Zeitraum unterbrochen ist.
Der Integer \ttt{format} speichert das Zeitformat, das in der Eingabedatei
auftriit. Die Kodierung ist dieselbe wie oben.

An dieser Stelle muss man erwähnen, dass kleinere Rechtecke im Programm als Zeiger zu 
den Objekten von \ttt{Rec} dargestellt werden.
Diese Entscheidung wurde aus diesem Grund getroffen, damit 
es nur eine Kopie eines Rechtecks in der Klasse \ttt{Solver} gibt und
damit jeder Prozess im Programm einen direkten Zugriff auf jedes Rechteck hat.

In der Klasse \ttt{Solver} gibt es einen \ttt{vector<Rec*>} namens \ttt{rectangles},
der alle Rechtecke aus diesem Beispiel speichert und der der Liste $Z$
aus dem \cref{sec:definitionen} entspricht.
Es gibt zwei \ttt{vector} von \ttt{vector<Rec*>} mit den Namen 
\ttt{rectangles\_stripes} und \ttt{unusedRectangles}. Die beiden 
haben die Größe \ttt{M} und der erste entspricht den Listen $S_j$ und
der zweite den Listen $U_j$.
Dazu gibt es einen \ttt{vector} von \ttt{set<Rec*>} der Größe \ttt{M} namens \ttt{placed},
in dem alle Rechtecke auftreten, die in den Streifen platziert sind.
Wir verwenden \ttt{set<Rec*>} aus dem Grund, dass die Rechtecke in diesem Container 
automatisch in der sortierten Reihenfolge der Koordinaten \ttt{x2} behalten werden.
Außerdem kann man Rechtecke in \ttt{set} sehr einfach einfügen.
Es gibt noch einen \ttt{vector} von \ttt{vector<Hole>} der Größe \ttt{M} namens \ttt{holes},
in dem alle Lücken aus jedem Streifen gespeichert werden. 
Dazu gibt es noch einen \ttt{vector} mit dem Namen \ttt{all\_holes}, in dem alle Lücken
aus allen Streifen hinzugefügt werden.

Die Methode \ttt{readFile()} liest die Daten aus der Textdatei ein.
Diese Methode nimmt als Argument einen String mit der Adresse der Datei in dem Rechner.
Das Eingabeformat in den Textdateien mit den Beispielen wurde geändert.
Die erste Zeile beinhaltet die Zahl $N$. Die zweite Zeile enthält
eine Folge von Öffnungszeiten des Flohmarkts. Die Uhrzeit mit einem geraden Index ist jeweils der
Beginn und die Uhrzeit mit einem ungeraden Index ist jeweils die Schließungszeit.
Die dritte Zeile enthält die Anzahl der Anmeldungen $n$ und es folgen $n$ Zeilen mit den Anmeldungen.
So wird zuerst die Zahl \ttt{N} eingelesen. 
Danach folgt ein String mit einer geraden Anzahl an Öffnugs-- und Schließungszeiten des Flohmarkts.
Dies wird mittels der Funktion \ttt{processInput()} eingelesen.
Dann wird die gesamte Pausenzeit des Flohmarkts mithilfe der Funktion \ttt{calculateRecess()} ermittelt
und in der Variable \ttt{recess} gespeichert. 
Dann werden die Variablen \ttt{M} und \ttt{START} bestimmt.
Es folgen danach die $n$ Anmeldungen,
die jeweils mittels der Funktion \ttt{timeToMinutes()} verarbeitet und
in \ttt{rectangles} gespeichert werden.

Im Programm gibt es die Methode \ttt{run()}, die den Lauf des Programms ausführt.
In dieser Funktion wird die Methode \ttt{distributeToStripes()} ausgeführt, die alle
Rechtecke aus \ttt{rectangles} auf die \ttt{vector}
\ttt{rectangles\_stripes} verteilt und die diese Listen danach unter Verwendung
von den Sortierkriterien \ttt{greaterEnd} einzeln sortiert.
Danach folgt der Lauf der Methode \ttt{processStripe()} für jeden Streifen.
Hier werden die Streifen im Greedy--Algorithmus verarbeitet.
Es wird durch jedes Rechteck \ttt{r} in jedem \ttt{vector} \ttt{rectangles\_stripes} iteriert.
Es wird geprüft, ob die Koordinaten des Rechtecks \ttt{r} $-1$ betragen und ob das Rechteck
in dem verarbeiteten Streifen beginnt.
Dann wird es für \ttt{r} nach einer Lücke im Streifen in \ttt{placed} gesucht.
Die Funktion \ttt{findNearestHole()} gibt die Koordinate \ttt{x1} für \ttt{r} aus,
wenn es Platz für dieses Rechteck gibt. Sonst wird $-1$ ausgegeben.
Wenn \ttt{x1} ausgegeben wird, wird dann auch \ttt{x2} gesetzt und 
das Rechteck \ttt{r} wird in alle Streifen in \ttt{placed} eingefügt, 
zu denen es gehört.

Nachdem die Methode \ttt{processStripe()} für alle Streifen ausgeführt worden ist,
wird der Gesamtflächeninhalt aller platzierten Rechtecke mittels der Methode 
\ttt{calculateAreaUsed()} bestimmt und es wird geprüft, ob 
die Koordinaten aller Rechtecke in \ttt{rectangles} größer sind als $-1$.
Wenn entweder der Gesamtflächeninhalt gleich dem Produkt $\ttt{N} \times \ttt{M}$ ist
oder wenn alle Rechtecke platziert wurden, wird das Programm an dieser Stelle abgebrochen.
Wenn nicht, wird die Methode \ttt{runOptimization()} laufen lassen, die
das Verbesserungsverfahren ausführt.


In der Methode \ttt{runOptimization()} gibt es eine Variable \ttt{area},
in der der Gesamtflächeninhalt einer Platzierung gespeichert wird, und 
eine boolsche Variable \ttt{result}, die dafür steht,
ob eine neue Platzierung vom Algorithmus angenommen wird.
Am Anfang hat \ttt{result} den Wert \ttt{true}, da das Ausgangsergebnis
angenommen werden muss.
Außerdem gibt es ein Objekt \ttt{hole} der Klasse \ttt{Hole}, das 
der aktuellen Lücke entspricht, in die ein nicht gelegtes Rechteck eingefügt wird.
Dazu gibt es ein Paar aus \ttt{Rec*} und \ttt{pair<int, int>} namens \ttt{rep}.
In dieser Variable wird ein Rechteck zum Einfügen in die Lücke \ttt{hole} gespeichert.
\ttt{pair<int, int>} enthält die Koordinaten der Lücke, in die \ttt{Rec*} eingefügt
werden soll.
Weiterhin gibt es zwei Iteratoren \ttt{itH} und \ttt{itR} wie im \cref{algo:verbesserung}.

Dann beginnt die Do--While--Schleife, die solange läuft, bis es eine Lücke gibt,
in die ein Rechteck eingefügt werden könnte.
In der Schleife gibt es eine Fallunterscheidung: Wenn \ttt{result} den Wert \ttt{true} hat und wenn
nicht.
Wenn die ursprüngliche Platzierung vom Algorithmus angenommen wurde,
werden zuerst der Wert \ttt{area} mittels der Funktion \ttt{calculateAreaUsed()}, 
die Listen \ttt{unusedRectangles} mittels der Methode \ttt{determineUnused()} und
die Liste \ttt{all\_holes} mittels der Funktion \ttt{findHoles()} bestimmt.
Da wir die Lücken und die nicht gelegten Rechtecke gerade bestimmten, 
werden \ttt{itH} und \ttt{itR} auf an die Anfänge der Liste
\ttt{all\_holes} und der Liste \ttt{unusedRectangles} im Streifen, zu
dem die Lücke unter dem Iterator \ttt{itH} gehört, gesetzt.
Dann wird die nächste Lücke \ttt{hole} mittels der Methode \ttt{findNextHole()} 
bestimmt und das nächste Rechteck \ttt{rep} mittels der Methode \ttt{findReplacement()}.
Bevor man \ttt{rep} in \ttt{hole} einfügt, muss man prüfen,
ob die Ausgabe der beiden genannten Funktionen richtig ist. 
Es kann beispielsweise sein, dass \ttt{hole} in so einem Streifen liegt, in dem alle Rechtecke 
platziert wurden und somit kann man kein Rechteck in \ttt{hole} einfügen.
Dazu ist die folgende While--Schleife.
Wenn mit dem Rechteck oder mit der Lücke etwas falsch ist, wird in \ttt{pair<int,int>} 
in \ttt{rep} ein negativer Wert ausgegeben. Der Wert $-1$ steht dafür, 
dass es keine Lücken mehr gibt, in die ein Rechteck eingefügt werden kann, also
wird an dieser Stelle die Do--While--Schleife abgebrochen.
Der Wert $-2$ steht dafür, dass der Iterator ans Ende von \ttt{unusedRectangles} in diesem
Streifen gelangt ist und dass der Iterator \ttt{itH} inkrementiert werden muss und eine 
neue Lücke gefunden werden muss.
Der Wert $-3$ bedeutet, dass das Rechteck in \ttt{hole} nicht eingefügt werden kann, 
da die Koordinate \ttt{x2} der Lücke kleiner ist als die Länge des Rechtecks \ttt{size}.
So würde das Recheck über die Grenzen des großen Rechtecks hinausreichen, wenn
es in diese Lücke platziert würde.
Diese While--Schleife wiederholt sich solange, bis eine Lücke gefunden wird,
in die man ein Rechteck platzieren kann. 
Dann wird die Methode \ttt{removeCollisions} ausgeführt, die 
die mit dem eingefügten Rechteck kollidierenden Rechtecke aus der Platzierung entfernt und 
in der eine neue Platzierung mit neuen Rechtecken generiert wird. 
Diese Methode gibt einen boolschen Wert aus, der besagt, ob die neue Platzierung besser
ist als die vorherige. Danach wiederholt sich die Do--While--Schleife.

Der Fall in der Do--While--Schleife für den Wert \ttt{result = false} ist analog 
zum ersten Fall, mit der Ausnahme, dass die Funktionen \ttt{calculateAreaUsed},
\ttt{determineUnused()} und \ttt{findHoles()} nicht ausgeführt werden.

Die Methode \ttt{findHoles()} findet alle Lücken
in allen Listen \ttt{placed}. 
Sie nimmt als Argument den Index des Streifens \ttt{p}.
Wenn \ttt{p = -1}, wird die Funktion auf allen Streifen ausgeführt. 
Es wird über jedes Rechteck im jedem Streifen iteriert und es wird geprüft,
ob der Wert \ttt{x2} eines Rechtecks mit dem Wert \ttt{x1} des darauffolgenden
Rechtecks übereinstimmt und ob das erste Rechteck im Streifen
den Wert \ttt{x1 = 0} und das letzte Rechteck im Streifen den Wert \ttt{x2 = N} besitzen.
Wenn nicht, wird eine Lücke mit den Koordinaten, die den Koordinaten 
der genannten Rechtecke entsprechen, und der Nummer des Streifens als ein Objekt
der Klasse \ttt{Hole} in die Liste \ttt{holes} zu diesem Streifen hinzugefügt.
Am Ende werden die Listen \ttt{holes} aus jedem Streifen zu einer gemeinsamen 
Liste \ttt{all\_holes} zusammengebracht und nach den Sortierkriterien 
\ttt{greaterHolesSize} sortiert.

Die Methode \ttt{findNextHole()} nimmt als Argument den Iterator \ttt{itH}.
In der Liste \ttt{all\_holes} wird geprüft, ob sich unter dem Iterator eine Lücke
befindet, zu der es ein Rechteck im Streifen, in dem sich die Lücke befindet,
in der Liste \ttt{unusedRectangles} gibt. Wenn nicht, wird \ttt{itH} inkrementiert,
bis so eine Lücke gefunden wird.
Wenn der Iterator zum Ende der Liste \ttt{all\_holes} gelangt, wird
eine Lücke mit Koordinaten $-1$ als Objekt der Klasse \ttt{Hole} ausgegeben,
was bedeutet, dass die Do--While--Schleife abgebrochen werden soll.

Die Methode \ttt{findReplacement()} nimmt zwei Argumente:
ein Objekt \ttt{hole} der Klasse \ttt{Hole}, also eine Lücke,
die von der Funktion \ttt{findNextHole()} ausgegeben wurde, und
den Iterator \ttt{itR}. 
In dieser Methode wird geprüft, ob \ttt{hole} den Wert $-1$ hat.
Wenn ja, wird ebenfalls der Wert $-1$ ausgegeben.
Danach wird geprüft, ob der Iterator \ttt{itR} das Ende der Liste
\ttt{unusedRectangles} in dem in \ttt{hole} angegebenen Streifen 
erreichte. Wenn ja, wird der Wert $-2$ ausgegeben.
Ansonsten wird eine Variable \ttt{rep} erstellt, die 
auf das Rechteck unter dem Iterator \ttt{itR} in \ttt{unusedRectangles}
gesetzt wird.
Anschließend wird geprüft, ob die Koordinate \ttt{x2} der Lücke \ttt{hole} 
nicht kleiner ist als die Größe \ttt{size} von \ttt{rep}.
Wenn doch, wird $-3$ ausgegeben. 
Sonst wird \ttt{rep} mit den Koordinaten der Lücke \ttt{x2} ausgegeben.

In der Methode \ttt{removeCollisions()} wird ein Paar bestehend
aus einem Rechteck und den Koordinaten einer Lücke als Argument genommen.
Es wird eine Kopie des ganzen \ttt{vector} \ttt{placed} gemacht
und \ttt{placedOld} genannt. 
Das Rechteck, das in die Lücke eingefügt wird, nennen wir \ttt{rr} und
die Lücke \ttt{hole}.
Dann werden die Koordinaten von \ttt{rr} gesetzt: \ttt{rr->x2 = hole.second} und
\ttt{rr->x1 = rr->x2 - getSize}.
Danach werden in \ttt{oldPlaced} alle Rechtecke gesucht, die mit
\ttt{rr} kollidieren, also wird es geprüft, ob es Rechtecke gibt, deren
Wert \ttt{x1} kleiner ist als der Wert \ttt{x2} des Rechtecks \ttt{rr}, 
wobei \ttt{rr} vor einem solchen Rechteck steht.
Alle kollidierenden Rechtecke werden in eine Liste \ttt{collisions}
hinzugefügt. 
Danach werden die Werte \ttt{x1} und \ttt{x2} aller Rechtecke, die zu \ttt{collisions}
gehören, zu $-1$ gesetzt.
Als Nächstes werden alle Rechtecke, deren Koordinaten $-1$ betragen, aus \ttt{placed}
entfernt.
Danach wird die Funktion \ttt{addNew} ausgeführt, die das Rechteck \ttt{rr} und ggf.
die neuen anderen Rechtecke in \ttt{placed} einfügt.
Sie gibt eine Liste mit Rechtecken aus, die neu platziert werden. Sie heißt \ttt{added}.
Danach wird der Gesamtflächeninhalt mittels der Funktion \ttt{calculateAreaUsed()}
bestimmt.
Wenn der neue Gesamtflächeninhalt größer ist als der vorherige,
wird die Methode \ttt{removeCollisions()} mit dem ausgegebenen Wert \ttt{true} abgebrochen.
Falls nicht, werden die Koordinaten
aller neu eingefügten Rechtecke aus der Liste \ttt{added}
zu $-1$ gesetzt und die 
ursprünglichen Koordinaten der kollidierenden Rechtecke werden zurückgesetzt. 
\ttt{placed} wird zu \ttt{oldPlaced} gesetzt, die Funktion gibt den Wert \ttt{false} 
aus und bricht ab.

Die Methode \ttt{addNew()} nimmt als Argument das Rechteck zum Einfügen in eine Lücke.
Es wird in dieser Funktion in \ttt{placed} an der richtigen Stelle platziert.
Dann lässt man die Funktion \ttt{processStripeReturn()},
die probiert, neue Rechtecke in \ttt{placed} zu platzieren, für jeden Streifen laufen. 
Die Methode \ttt{placed} funktioniert ähnlich wie die Methode \ttt{processStripe}.
Es wird für jeden Streifen die Liste \ttt{rectangles\_stripes} iteriert und bei jedem
iterierten Rechteck \ttt{r} wird geprüft, ob er mit dem verarbeiteten Streifen beginnt und
ob er nicht platziert wurde.
Dann wird mittels der Funktion \ttt{findNearestHole()} die nächste Lücke in diesem Streifen
in \ttt{placed} gesucht.
Wenn es so eine Lücke gibt, wird geprüft, ob die Lücken in allen Streifen in \ttt{placed}
bestehen, zu denen das iterierte Rechteck gehört.
Wenn ja, wird dieses Rechteck in alle diesen Streifen in \ttt{placed} eingefügt.
Diese Methode gibt eine Liste \ttt{added} aus, in der sich alle neu eingefügten Rechtecke 
befinden.
In der Methode \ttt{addNew} wird die Liste \ttt{added} aus
allen Streifen zusammengebracht und ausgegeben.


