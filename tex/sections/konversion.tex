\subsubsection{Konversion der Eingabe}
Wie schon im \cref{sec:definitionen} erwähnt wurde, können die Voraussetzungen bezüglich des
\fp s auf andere Größen übertragen werden. Da die Größen des Rechtecks $R$ sowie des Zeitraums fest sind
und auf 1000 Metern bzw. auf den Zeitraum von 8:00 is 18:00 beschränkt sind, konvertieren wir
die Eingabe, indem wir den 
Beginn $B$ vom Ende $E$ subtrahieren und den Beginn des Zeitraumes auf 0 setzen.
So bleibt auch der Wert $M$, also die Differenz von $E$ und $B$, gleich. 
Analog müssen wir die Eingabe für die kleineren Rechtecke $r_i$ entsprechend konvertieren, indem wir
von jedem $b_i$ und $e_i$ den Wert $B$ abziehen. Für die Aufgabe selbst hat diese Konversion keine
Bedeutung und funktioniert auch, wenn ein angegebener Zeitraum sich vom ursprünglichen Zeitraum unterscheidet.

Mit dieser Konversion können wir ebenfalls mehrtägige Flohmärkte behandeln.
Zur Darstellung eines mehrtägigen Flohmarktes kann man die gesamten Öffnungszeiten des Flohmarkts in Stunden angeben,
z. B. der Zeitraum eines Flohmarkts, der zwei Tage von 10:00 bis 17:00 dauert,
kann als von 10:00 bis 17:00 und von 34:00 bis 41:00 dargestellt werden. 
%Dann ist der Zeitraum von 17:00 bis 34:00 an keiner Stelle besetzt.
Ebenfalls, wenn der Zeitraum eines eintägigen Flohmarkts an einer Stelle unterbrochen ist,
etwa dauert der Flohmarkt von 7:00 bis 9:00 und dann von 12:00 bis 15:00, können die 
beiden Zeiträume in einem Beispiel angegeben werden. Wir können dann diese Instanz lösen,
indem wir den ganzen Zeitraum von 7:00 bis 15:00 betrachten und in der Pausenzeit
wird keine Anmeldung angenommen.
Mehrere unterschiedlichen Flohmärkte mit derselben Länge $N$ kann man analog kodieren.
Beispielsweise gründet man eine Firma, die Flohmärkte an vielen Orten organisiert, aber
die Firma verfügt nur über Stände mit einer festen Länge. Dann muss man die Eingabe nur
entsprechend kodieren.
%Es hängt nur von der Eingabe ab.


Außerdem werden im ursprünglichen Problem alle Zeiten in vollständigen Stunden angegeben.
Diese Aufgabe kann sehr leicht zu Zeiten in Minuten ergänzt werden.
Das lohnt sich vor allem dann, wenn die Öffnungszeit zur halben Stunde fällt.
Dazu konvertiert man die Eingabe
am Anfang auf folgende Weise: Man kann einfach alle Zeiten zu Minuten umrechnen, indem
man vollständige Stunden mal 60 multipliziert.
Obwohl die weiteren Betrachtungen sich grundsätzlich auf vollständige Stunden beziehen
(wie in der Aufgabenstellung),
soll man nicht vergessen, dass alle diesen Gedanken sich auf Minuten übertragen lassen.