\subsection{Formulierung des Problems}\label{sec:definitionen}
Gegeben sei eine Strecke der Länge $N$ und ein Zeitraum von $B$ bis $E$.
Außerdem gegeben sei eine Liste von $Z$ Anmeldungen. 
Die Anmeldungen betreffen die Vermietung eines Teils der Strecke in einer konkreten Zeitspanne.
Jede \textbf{\textit{Anmeldung}} $i$ besteht aus einer Strecke $s_i$ ($0 < s_i \leqslant N$),
einem Mietbeginn $b_i$ ($B \leqslant b_i < E$) und einem Mietende $e_i$ ($b_i < e_i \leqslant E$).
In diesem Problem werden Strecken in vollständigen Metern behandelt 
und alle Zeiten werden in vollständigen Stunden angegeben.
Obwohl $N$ auf 1000 Meter, $B$ auf 8:00 und $E$ auf 18:00 in der ursprünglichen Aufgabe festgelegt sind,
kann die folgende Lösungsidee auf beliebige Größen, die die Aufgabenbedingungen erfüllen, übertragen werden.
Das gelieferte Programm kann somit mit unterschiedlichen Werten umgehen.

Die Aufgabe ist ein Optimierungsproblem.
Man soll so eine Teilfolge von $k$ Anmeldugen wählen,
dass alle gewählten Strecken in den angebenen Zeiten vermietet werden können, d.h.,
für jede Anmeldung eine freie stetige Strecke der angegbenen Länge 
in der angegebenen Zeitspanne durchgehend zur Verfügung steht,
und dazu die Mieteinnahmen möglichst hoch sind, wobei der Preis 1 Euro pro Meter pro Stunde beträgt.

Man kann das Problem auf folgende Weise modellieren. 
Wir setzen: $M \coloneqq E - B$.
Wir bilden ein \textit{\textbf{großes Rechteck}} $R$ der Größe $N \times M$.
So kann man analog jede Anmeldung $i$ als ein \textit{\textbf{kleineres Rechteck}}
$r_i$ der Größe $s_i \times m_i$ darstellen, wobei $m_i \coloneqq e_i - b_i$.

So können wir die obige Aufgabe umformulieren:
Wähle so eine Teilfolge $Z'$ von Rechtecken aus $Z$,
die eine Anordnung innerhalb von $R$ bilden,
dass kein Paar der Rechtecke in $Z'$ sich überdeckt und
der Gesamtflächeninhalt aller Rechtecke in $Z'$ maximal ist.
Als Fläche eines kleineren Rechtecks $r_i$ bezeichnen wir das Produkt $m_i \times s_i$.

Genauer gesagt: Wenn wir das große Rechteck und die kleineren platzierten Rechtecken
an einem Koordinatensystem abbilden,
besitzt jedes Rechteck $r_i$ in $Z'$ 4 Ecken,
die den folgenden Punkten entsprechen:
$(x_i, b_i), (x_i, e_i), (x_i + s_i, e_i), (x_i + s_i, b_i)$.
Man beachte, dass $b_i$, $e_i$ und $s_i$ fixiert sind. 
So ist die Aufgabe, nur $x_i$ so zu wählen, dass die Bedingungen der Aufgabe erfüllt werden.
Wir können uns dieses Problem so vorstellen, dass die Länge $s_i$ und die Breite $m_i$
jedes Rechtecks $r_i$, sowie seine Anordnung entlang der $y$--Achse fixiert sind,
und wir das Rechteck nur entlang der $x$--Achse zwischen den $x$--Werten von 0 und $N-s_i$ verschieben können.

In den weiteren Betrachtungen nennen wir unsere Aufgabe das \fp.