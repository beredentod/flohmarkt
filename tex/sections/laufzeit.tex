\subsection{Laufzeit}
\begin{itemize}
	\item $M$ --- die Breite des großen Rechtecks $R$, die Anzahl der Streifen 
	\item $n$ --- $|Z|$, also die Anzahl der kleineren Rechtecke
\end{itemize}

Die Größe $M$ in der Aufgabe tritt in vollständigen Stunden auf.
Wenn die Eingabe zu Minuten konvertiert wird, wird diese Variable
in vollständigen Minuten betrachtet.
Im \cref{sec:greedy} wird beschrieben,
dass die Breite jedes Streifens 1 Stunde bzw. 1 Minute entspricht.
Somit kann man die Größe $M$ auch als die Anzahl der Streifen betrachten.


\begin{itemize}
	\item Vorbereitung der Eingabe: $O(M \cdot n \log n)$ (worst--case)
	\begin{itemize}
		\item Einlesen aller Rechtecke und Erstellung der Liste $Z$: $O(n)$

		\item Erstellung von Listen \ttt{placed}, \ttt{unusedRectangles} und \ttt{holes}
		für jeden Streifen (s. \nameref{sec:umsetzung}): $O(M)$

		\item Verteilung jedes Rechtecks auf die Listen $S_j$, zu denen sie gehören: $O(M \cdot n)$ (worst--case)\\
		Im schlimmsten Fall gehört jedes Rechteck zu jeder Liste $S_j$, wenn jede Anmeldung
		den ganzen Zeitraum eines Flohmarkts betrifft.
		So muss jedes Rechteck jeder Liste $S_j$ hinzugefügt werden.

		\item Sortierung der Listen $S_j$: $O(M \cdot n \log n)$ (worst--case)\\
		Es gibt $M$ Listen und in jeder Liste kann es im schlimmsten Fall
		alle Rechtecke geben. Die linear--logarithmische Laufzeit
		ist durch das Sortieren verursacht.
	\end{itemize}

	\item Der Greedy--Algorithmus am Anfang: $O(n(n + M \log n))$ (worst--case, amortisierte Laufzeit)\\
	Obwohl man die Funktion zur Verarbeitung eines Streifens $M$--mal 
	laufen lässt, kann eine Laufzeitanalyse pro Lauf dieser Funktion zu pessimistisch sein.
	Es ist unmöglich, dass ein Platz für $n$ Rechtecke in jedem Streifen $M$--mal gesucht wird,
	da wir voraussetzen, dass jedes Rechteck $r_i$ nur im Streifen $b_i$ gelegt werden kann
	und außerdem können $n$ Rechtecke nicht $M$--mal platziert werden, da jedes Rechteck
	nur einmal gelegt werden kann.
	Stattdessen analysieren wir die amortisierte Laufzeit für das Platzieren jedes Rechtecks allein,
	deshalb wird die endliche Laufzeit mit dem Faktor $n$ multipliziert, da man im schlimmsten Fall
	alle $n$ Rechtecke in $R$ legen muss.

	\begin{itemize}
		\item Das Finden der passenden, freien $x$--Koordinaten: $O(n)$ (worst--case)\\
		Im schlimmsten Fall muss man in einem Streifen über $n-1$ Rechtecke iterieren,
		um einen freien Platz für ein Rechteck zu finden.

		\item Das Finden der genauen Stelle in den restlichen Streifen: $O(M \log n)$ (worst--case)\\
		Nicht in jedem Streifen müssen sich dieselben Rechtecke befinden und ein Rechteck 
		kann zu mehreren Steifen gehören.
		In jedem Streifen muss man die genaue Position zum Platzieren des Rechtecks finden.
		Da man im Programm \ttt{set} verwendet, sind alle Rechtecke im Streifen
		immer in aufsteigender Reihenfolge ihrer Koordinaten $x_i$. 
		So kann ein Rechteck mittels der eingebauten Funktion \ttt{insert} 
		in \ttt{set} eingefügt werden. Die Einfügen--Operation erfolgt in logarithmischer Laufzeit
		bezüglich der Anzahl der Rechtecke im Streifen: $O(\log n)$.\footnote{
		\href{https://en.cppreference.com/w/cpp/container/set/insert}{https://en.cppreference.com/w/cpp/container/set/insert}}
		Im schlimmsten Fall gehört ein Rechteck zu allen Streifen, deshalb muss die endliche Laufzeit
		mal $M$ multipliziert werden: $O(M \log n)$.

		%Das erfolgt mittels der eingebauten Funktion \ttt{upper\_bound}, die in C++ in $O(\log n)$ läuft.\footnote{\href{https://en.cppreference.com/w/cpp/algorithm/upper_bound}{https://en.cppreference.com/w/cpp/algorithm/upper\_bound}}.

	\end{itemize}

	\item Ein Lauf des Verbesserungsalgorithmus pro Paar Lücke--Rechteck: $O(n(n + M \log n))$ (worst--case, amortisierte Laufzeit)

	\begin{itemize}
		\item Berechnung des Gesamtflächeninhalts aller platzierten Rechtecke: $O(n)$ (worst--case)\\
		Im schlimmsten Fall können alle Rechtecke ins große Rechteck $R$ gelegt werden.

		\item Bestimmung aller nicht gelegten Rechtecke: $O(M \cdot n \log n)$ (worst--case)\\
		Im schlimmsten Fall gibt es ein gelegtes Rechteck, das genauso groß ist wie $R$, und somit
		alle $n-1$ restlichen kleineren Rechtecke zu Listen $U_j$ gehören. 
		Im schlimmsten Fall können alle diese restlichen Rechtecke zu allen Listen $U_j$ gehören.
		Die linear--logarithmische Laufzeit ist dem Sortieren der Listen $U_j$ geschuldet.

		\item Auffinden aller Lücken: $O(M \cdot n \log n)$ (worst--case)\\
		Im schlimmsten Fall gibt es in jedem Streifen $n+1$ Lücken, wenn kein Paar
		der Rechtecke in demselben Streifen eine gemeinsame Seite hat ---
		dann gibt es Lücken auf beiden Seiten
		jedes Rechtecks. Dazu kann es im schlimmsten Fall $n$ Rechtecke in jedem Streifen geben.
		Die linear--logarithmische Laufzeit ist mit dem Sortieren der Liste $H$ verbunden.

		\item Entfernung aller kollidierenden Rechtecke: $O(M \cdot n)$ (worst--case)\\
		Beim Einfügen eines Rechtecks $r$ in eine Lücke muss man in allen Streifen, zu denen $r$
		gehört, prüfen, ob es Kollisionen gibt.
		Im schlimmsten Fall gehört ein Rechteck zu allen $M$ Streifen und man muss in allen
		Streifen $n-1$ Rechtecke entfernen.

		\item Einfügen neuer Rechtecke in neue Lücken: $O(n(n + M \log n))$
		(worst--case, amortisierte Laufzeit)\\
		In diesem Teil lohnt es sich mehr, eine amortisierte Laufzeitanalyse pro Rechteck durchzuführen.
		Im schlimmsten Fall muss man $n-1$ Rechtecke ins Rechteck $R$ einfügen, deshalb
		wird die endliche Laufzeit mit dem Faktor $n$ multipliziert.
		Im schlimmsten Fall gehört jedes Rechteck zu jedem der $M$ Streifen.
		Beim Einfügen jedes Rechtecks $r$ muss man zuerst die potenziellen Koordinaten
		für $r$ im Streifen $b_r$ finden.
		Dazu kann man im schlimmsten Fall über $n-1$ Rechtecke iterieren: $O(n)$.
		Dann muss man in jedem Streifen, zu dem $r$ gehört, 
		prüfen, ob es genug Platz dafür gibt.
		Das erfolgt mittels der eingebauten Funktion \ttt{upper\_bound},
		die in logarithmischer Zeit bezüglich der Anzahl der Rechtecke im
		Streifen läuft: $O(\log n)$.\footnote{\href{https://en.cppreference.com/w/cpp/algorithm/upper_bound}{https://en.cppreference.com/w/cpp/algorithm/upper\_bound}}
		%Das wird wieder mittels der Funktion \ttt{upper\_bound} ermittelt,
		%die in logarithmischer Zeit läuft.
		Dann wird das Rechteck allen Streifen hinzugefügt, zu denen es gehört.
		Diese Operation erfolgt mithilfe der Funktion \ttt{insert}, die in
		logarithmischer Zeit läuft: $O(M \log n)$.
		Somit ergibt sich pro Rechteck $r$ die Laufzeit: $O(n + M \log n)$.

		\item Zurücksetzung der ursprünglichen Platzierung: $O(M \cdot n)$ (worst--case)\\
		Wenn eine Platzierung einen niedrigeren Flächeninhalt besitzt als 
		die usprüngliche Anordnung, wird diese Platzierung vom Algorithmus abgelehnt
		und die ursprünliche Anordnung wird zurückgesetzt.
		Dazu muss man den Inhalt aus allen $M$ Streifen kopieren, wobei sich in jedem Streifen höchstens 
		$n$ Rechtecke befinden können.
	\end{itemize}

\end{itemize}

Wie man im \cref{algo:verbesserung} erkennt, unterscheidet man zwischen zwei möglichen
Läufen des Verbesserungsalgorithmus in der While--Schleife:
Wenn ein neues Ergebnis angenommen wird und falls nicht.
Wenn ein Ergebnis angenommen wird, muss man zusätzlich die Funktionen zur Bestimmung aller Lücken und aller nicht gelegten
Rechtecke laufen lassen.
Wie wir in den Betrachtungen zur Laufzeitanalyse zum Verbesserungsalgorithmus pro ein Paar Lücke--Rechteck 
sehen, besitzen diese zwei Funktionen niedrigere
Laufzeiten als die Gesamtlaufzeit der restlichen Prozesse, deshalb unterscheiden wir
nicht zwischen den Laufzeiten der While--Schleife.
%, in denen ein Ergebnis 
%angenommen wird, und denjenigen, in denen das nicht der Fall ist.
Somit ergibt sich die folgende Laufzeit für ein Paar von einer Lücke $L$ und einem Rechteck $r$,
wobei $r$ in $L$ eingefügt werden soll: $O(n(n + M \log n))$ (worst--case).

Nun bestimmen wir die Anzahl an Paaren Lücke--Rechteck, die vom Verbesserungsalgorithmus
verarbeitet werden. Die Funktion zur Bestimmung aller nicht gelegten Rechtecke kann
höchstens $n-1$ Rechtecke und die Funktion zum Auffinden aller Lücken kann 
höchstens $M \cdot n$ Lücken finden.
Allerdings, wenn $n-1$ Rechtecke nicht gelegt sind, gibt es höchstens $2M$ Lücken.
Wenn die Anzahl der Lücken wächst, sinkt die Anzahl der nicht gelegten Rechtecke.
Deshalb gibt es am meisten Möglichkeiten, wenn die beiden Anzahlen halbiert werden.
So gibt es maximal $O(M \cdot n/2 \cdot n/2) = O(M \cdot n^2)$ Möglichkeiten.

Im schlimmsten Fall muss man alle diesen Kombinationen durchgehen, bis man
zu einem Ergebnis gelangt, das vom Verbesserungsalgorithmus angenommen wird.
So legen wir fest, die Laufzeit des Verbesserungsalgorithmus
für jede Kombination der Lücken und Rechtecke aus einem Paar von Listen $H$ und $U_j$
beträgt im worst--case:
\[
	O(n(n + M \log n) \cdot M \cdot n^2) = O((n^2 + M \cdot n \log n) \cdot M \cdot n^2)
	= O(M \cdot n^4 + M^2 \cdot n^3 \log n).
\]

Es bleibt noch die Abschätzung, wie viel mal man die Listen $H$ und $U_j$ bestimmt.
Allgemein kann man die Laufzeit eines Bergsteigeralgorithmus schwer abschätzen.
Allerdings ist ein Ergebnis auf den Flächeninhalt des großen Rechtecks $R$ beschränkt
und die Einheiten in dieser Aufgabe sind ganzzahlig (vollständige Meter, vollständige Stunden/Minuten).
So können wir feststellen, dass die Ergebnisse des Verbesserungsalgorithmus mit jeder
neuen Bestimmung der Listen $H$ und $U_j$ eine streng monoton wachsende Funktion bilden.
Es gibt also eine feste Anzahl an möglichen Verbesserungen.
Theoretisch könnte es einen Fall geben, in dem das Anfangsergebnis nach dem Lauf
des Greedy--Algorithmus 1 beträgt und in dem dieses Ergebnis mit jeder 
neuen Bestimmung der Listen $H$ und $U_j$ um 1 verbessert wird,
aber diese Situation ist aus praktischer Sicht kaum vorstellbar. 
Deshalb führen wir eine Variable $k$ ein, deren Wert zwischen 1 und $N \times M$ liegt, die
dafür steht, wie oft die Listen $H$ und $U_j$ bestimmt werden müssen.
In den Beispielen \hyperref[ex:1]{1}--\hyperref[ex:7]{7} überschreitet $k$ 10 nicht.
So beträgt die finale Laufzeit im worst--case:
\[
	O(k(M \cdot n^4 + M^2 \cdot n^3 \log n)).
\]
