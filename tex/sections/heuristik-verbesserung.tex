\subsubsection{Heuristisches Verbesserungsverfahren}
Wir probieren, das Ausgangsergebnis zu verbessern.
Bezeichnen wir ab jetzt ein beliebiges Ergebnis, also eine beliebige Anordnung
der kleineren Rechtecke innerhalb des großen Rechtecks $R$, die unser Programm liefert, 
als $C$. Insbesondere nennen wir unser Ausgangsergebnis $C_A$.


Man leicht feststellen, dass man mithilfe des obengenannten Greedy--Algorithmus 
das Ausgangsergebnis nicht optimieren kann. Wir haben begründet, dass dieser Algorithmus
an jeder Stelle stets die aktuell optimale Variante wählt. 
Außerdem dürfen wir diesen Algorithmus nicht nochmal nutzen,
da wir voraussetzen, dass die Streifen in der aufsteigender Reihenfolge ein nach dem anderen
verarbeitet werden. Dann kann es sein, dass es sich eine Lücke zwischen den Punkten $(x_j, j)$ und 
$(x_j + \ell, j)$ der Länge $\ell$ an einer Stelle in einem Streifen $j$ befindet
und dass ein Rechteck $r_i$ mit $s_i < \ell$ theorethisch hineinpassen würde, aber
es ist nicht mehr gesichert, dass es die Lücken direkt darüber in oberen Streifen $j+1, j+2, ...$
geben würde.


Deshalb führen wir ein neues Verfahren ein. 
Sei $C$ eine beliebige Platzierung von Rechtecken innerhalb von $R$.
Nennen wir $C$ \textit{das aktuelle Ergebnis}. 
Die allgemeine Idee des Verbesserungsverfahrens besteht darin,
man findet eine Lücke in einem Streifen 
und man legt ein noch nicht platziertes Rechteck $r$ in die Lücke, gegebenfalls
muss man die Rechtecken, die mit $r$ kollidieren, aus der Platzierung entfernen
und somit entstehen neue Lücken, die mit anderen nicht gelegten Rechtecken gefüllt werden
können.
So kommt man auf ein neues Ergebnis $C'$.
Wir vergleichen die Gesamtflächeninhalte der kleineren Rechtecke innerhalb von $R$
der Platzierungen $C$ und $C'$. 
Wenn $C'$ größer ist als $C$, wird $C'$ das aktuelle Ergebnis und der Vorgang
wiederholt sich. Somit stellt dieses Verfahren das
heuristisches Verfahren, das als ein Bergsteigeralgorithmus klassifiziert werden kann.
