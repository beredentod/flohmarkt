\subsubsection{Heuristisches Verbesserungsverfahren}\label{sec:verbesserung}
Man kann leicht feststellen, dass, wenn alle Rechtecke aus $Z$ im Laufe des
Greedy--Algorithmus in $R$ platziert wurden oder wenn die ganze Fläche von $R$ bedeckt wurde, 
das Problem für diese Eingabe optimal gelöst wurde.
Allerdings lässt sich nicht nachweisen, dass der vorgestellte Algorithmus 
stets eine optimale Platzierung liefert.
Hingegen kann man sogar festellen, dass es bessere Ergebnisse gibt als die,
die am Anfang geliefert werden. 


Wir probieren, das Ausgangsergebnis heuristisch zu verbessern.
Bezeichnen wir ab jetzt ein beliebiges Ergebnis, also eine beliebige Anordnung
der kleineren Rechtecke innerhalb des großen Rechtecks $R$, die unser Programm liefert, 
als $C$. Insbesondere nennen wir unser Ausgangsergebnis $C_A$.


Offensischtlich kann man mithilfe des obengenannten Greedy--Algorithmus 
das Ausgangsergebnis nicht optimieren. Wir haben begründet, dass dieser Algorithmus
an jeder Stelle stets die aktuell optimale Variante wählt. 
Außerdem dürfen wir diesen Algorithmus nicht nochmals nutzen,
da wir voraussetzen, dass die Streifen in der aufsteigender Reihenfolge einer nach dem anderen
verarbeitet werden. Dann kann es sein, dass sich eine Lücke der Länge $\ell$
zwischen zwei Punkten $(x_j, j)$ und 
$(x_j + \ell, j)$ in einem Streifen $j$ befindet
und dass ein Rechteck $r_i$ mit $s_i \leqslant \ell$ theoretisch hineinpassen würde, aber
es ist nicht mehr gesichert, dass es die Lücken direkt darüber in oberen Streifen $j+1, j+2, ...$
geben würde.


Deshalb führen wir ein neues Verfahren ein. 
Sei $C$ eine beliebige Platzierung von Rechtecken innerhalb von $R$.
Nennen wir $C$ \textit{das aktuelle Ergebnis}. 
Die allgemeine Idee lautet:
Ein nicht platziertes Rechteck $r$ wird in $R$ platziert, ggf. 
alle Rechtecke, die mit $r$ kollidieren, werden aus $R$ entfernt und
die neu entstandenen Lücken werden mit nicht gelegten Rechtecken gefüllt. 
So kommt man auf einen neuen Zustand $C'$, also eine neue Platzierung der Rechtecke.
Es wird dann überprüft, ob der Gesamtflächeninhalt aller platzierten Rechtecke in der Platzierung $C'$
größer ist als der in der Platzierung $C$.
Wenn ja, wird $C'$ das aktuelle Ergebnis und der Vorgang wiederholt sich,
bis es nicht mehr möglich ist, einen Zustand $C$ weiter zu verändern.
Wir sehen, dieser Ansatz nimmt viel Inspiration aus der Idee der Bersteigeralgorithmen,
dennoch lässt er sich schlecht als einer klassifizieren.
Unser Algorithmus ist zu deterministisch,
es fehlen starke Mutationen in seinen Lauf, die das Ergebnis deutlich beinflussen könnten,
und es fehlen Anwendungen der Randomisierung.


Wie kommt es zur Veränderung der Platzierung und wann bestimmen wir,
dass es unmöglich ist, einen Zustand weiter zu verändern?
Unser Verbesserungsverfahren arbeitet mit Lücken, die 
nach der Platzierung $C_A$ entstehen.
Die Idee ist, man findet eine Lücke in einem Streifen 
und man legt ein noch nicht platziertes Rechteck $r$ in die Lücke,
ggf. muss man die Rechtecke,
die mit $r$ kollidieren, aus der Platzierung entfernen
und somit entstehen neue Lücken,
die mit anderen nicht gelegten Rechtecken gefüllt werden können.
So kommt man auf ein neues Ergebnis.
Man hört auf, wenn es keine Lücken mehr gibt, für die ein Rechteck zum Platzieren zur Verfügung steht.


Zuerst muss man die nicht platzierten Rechtecke für jeden Streifen bestimmen. 
So legen wir für jeden Streifen $j$ eine Liste $U_j$ fest, in der sich alle 
Rechtecke aus dem Streifen $j$ befinden, die nicht platziert sind.
Die Liste $U_j$ muss man auf eine bestimmte Weise ordnen.
Jedes Rechteck $r_i$ in jeder solchen Liste sortieren wir nach diesen Kriterien:
1) aufsteigend nach der Länge $s_i$ und 2) aufsteigend nach dem Beginn $b_i$.
Die Entscheidung, diese Sortierkriterien zu wählen, ergibt sich experimentell 
und wird im \cref{sec:diskussion-ergebnisse} diskutiert.


Danach muss man die Lücken in jedem Streifen in einer Platzierung finden.
So legen wir eine Liste $H_j$ für jeden Streifen $j$ fest, in der sich alle
Lücken aus diesem Streifen befinden. Man findet sie, indem
man durch jedes im Streifen $j$ platziertes Rechteck $r_i$ iteriert und jeweils überprüft,
ob das Rechteck $r_i$ mit dem Rechteck $r_{i+1}$ eine gemeinsame Seite haben. Wenn nicht, gibt es eine Lücke
zwischen den Rechtecken $r_i$ und $r_{i+1}$.
Dazu muss man auch untersuchen, ob das erste Rechtecks 
und das letzte Rechteck im Streifen direkt an den Seiten des großen Rechtecks liegen. 
Wenn nicht, enstehen auch Lücken zwischen den Rechtecken und den Seiten des großen Rechtecks $R$.  

Danach werden alle Listen $H_j$ zu einer Liste $H$ zusammengebracht.
Diese Liste muss auch auf eine bestimmte Weise geordnet sein. 
Experimentell ergeben sich die folgenden Sortierkriterien für jede Lücke $L_i$: 
1) fallend nach der Größe der Lücke $L_i$ und 2) aufsteigend nach dem
Index des Streifens. Diese Entscheidung wird ebenfalls im \cref{sec:diskussion-ergebnisse}
diskutiert.

\begin{algorithm}[h]
\caption{Das heuristische Verbesserungsvefahren}
\label{algo:verbesserung}
\flushleft
\textbf{Eingabe:} $R$ --- das große Rechteck, $Z$ --- die Liste mit allen kleineren Rechtecken.
\begin{algorithmic}[1]
\State $C_A \gets$ Greedy($R, Z$)
\State $H \gets$ getAllHoles$(C_A)$
\State $U \gets$ getRecs$(C_A)$
\State $C \gets C_A$
\State $itH \gets H.begin$
\State $j \gets itH.$Streifen\Comment{$j$ ist der Index des Streifens, in dem sich die Lücke zum Iterator $itH$ befindet}
\State $itR \gets U_j.begin$
\While{$itH \neq H.end$}\label{line:abbruch}
	\State $\Sigma_C \gets |C|$ \Comment{der Gesamtflächeninhalt aller platzierten Rechtecke in $C$}
	\State $C' \gets$ next$(C, itH, itR)$
	\State $itR \gets itR + 1$
	\State $\Sigma_{C'} \gets |C'|$
	\If {$\Sigma_{C'} > \Sigma_C$} \label{line:vergleich}
		\State $H \gets$ getAllHoles$(C')$
		\State $U \gets$ getRecs$(C')$
		\State $itH \gets H.begin$
		\State $k \gets itH.$Streifen
		\State $itR \gets U_j.begin$
		\State $C \gets C'$
	\EndIf
\EndWhile
\end{algorithmic}
\end{algorithm}

Der \cref{algo:verbesserung} zeigt eine vereinfachte Vorgehensweise des Verbesserungsverfahrens.
Die Funktion $getAllHoles(C)$ findet alle Lücken in allen Streifen im Rechteck $R$ in der
aktuellen Platzierung $C$ und bestimmt die Liste $H$.
Die Funktion $getRecs(C)$ findet alle nicht platzierten Rechtecke und verteilt sie auf
die Listen $U_j$ für jeden Streifen $j$.
Die Funktion $next(C, itH, itR)$ besteht in diesem Algorithmus selbst aus drei Funktionen,
die im Folgenden beschrieben werden.

Die erste Funktion bestimmt die nächste Lücke, in die ein nicht platziertes Rechteck 
eingefügt wird. Es gibt einen Iterator $itH$, der am Anfang am Beginn der Liste $H$ gesetzt wird.
Im Laufe des Algorithmus bewegt sich der Iterator und zeigt auf nächste Lücken.
Eine Lücke bezeichnen wir als \textit{geeignet}, wenn sie sich in einem
Streifen $j$ befindet, zu dem die entsprechende Liste $U_j$ nicht leer ist, d.h., 
es gibt mindestens ein Rechteck, das in diese Lücke eingefügt werden kann.
Somit bewegt sich der Iterator in der Liste $H$ und wenn er auf eine geeignete Lücke 
stößt, wird diese Lücke durch die folgenden zwei Funktionen bearbeitet.
Insbesondere beachte man, dass die While--Schleife in \cref{line:abbruch} abbricht, wenn
der Iterator $itH$ zum Ende der Liste $H$ gelangt, also dann,
wenn es keine geeigneten Lücken mehr gibt. 

Die nächste Funktion wählt ein Rechteck $r$, das in eine gewählte geeignete Lücke $L$
eingefügt wird. Da die gewählte Lücke sich in einem Streifen $j$ befindet, stammt $r$
aus der Liste $U_j$.
Es gibt auch einen internen Iterator $itR$ für die Liste $U_j$,
der bei jeder neuen Lücke an den Anfang der Liste gesetzt wird.
Wenn ein Rechteck $r$ in einem Lauf $t$ der While--Schleife in $L$ eingefügt wird,
wird $itR$ danach inkrementiert und im darauf folgenden Lauf der 
Schleife $t+1$ wird ein unterschiedliches Rechteck in die Lücke $L$ gelegt.
Wenn der Iterator $itR$ bis zum Ende der Liste $U_j$ gelangt, wird der Iterator $itH$
in der Liste $H$ inkrementiert und somit eine neue 
geeignete Lücke gesucht.

Die letzte Funktion nimmt das unter dem Iterator $itR$ stehende Rechteck $r$
und legt es in die unter dem Iterator $itH$ stehende Lücke $L$.
Diese Funktion bereitet eine neue Platzierung $C'$ vor.
Seien die Koordinaten der Lücke $x_1$ und $x_2$ und der Streifen, in dem sich $L$ befindet, sei $j$. 
Seien $x_r$ und $x_r + s_r$ die $x$--Koordinaten von $r$.
Das Rechteck $r$ wird in $R$ so gelegt, dass $x_2 = x_r + s_r$. Auf diese Weise beträgt
der Wert $x_r \coloneqq x_2 - s_r$.\footnote{Die Situation, in der $x_r < 0$ gilt, wird  in der
zweiten Funktion ausgeschlossen. Wenn es dazu käme, wird der Iterator $itR$ inkrementiert wird und das nächste Rechteck gewählt.}
Selbstverständlich kann es an dieser Stelle zu Kollisionen kommen --- Rechtecke können sich überdecken.
Vor dem Platzieren prüft man nicht, ob es durchgehend eine Lücke zwischen den Werten $x_r$ und $x_r + s_r$ 
in allen Streifen $k$ gibt, wobei $b_r \leqslant k < e_r$.
Deshalb entfernt man nun alle Rechtecke, die mit $r$ kollidieren, also all diejnigen, die 
zumindest zum Teil zwischen $x_r$ und $x_r + s_r$ in einem Streifen $k$ liegen.
So entstehen auch neue Lücken, deshalb versuchen wir an dieser Stelle 
die Lücken mit anderen Recktecken zu füllen.
Dazu verarbeiten wir alle Streifen $k$ ($b_r \leqslant k < e_r$), indem wir
jedes in $C'$ noch nicht gelegte Rechteck $r_i$
in jeder Liste $S_k$ (in der ursrpünglichen Reihenfolge) untersuchen.
Wie beim Greedy--Algorithmus am Anfang versuchen wir, ein Rechteck
$r_i$ im Streifen $g$ zu legen nur, wenn $g = b_i$
(und wenn es eine Lücke zwischen $x_{r_i}$ und $x_{r_i} + s_i$ gibt).
Allerdings müssen wir die Streifen $b_i+1, b_i+2, ..., e_i-1$ vor dem Platzieren prüfen,
ob es in ihnen durchgehend Lücken zwischen $x_{r_i}$ und $x_{r_i} + s_i$ gibt. 
Nur, wenn in allen Streifen $b_i, b_i + 1, ..., e_i-1$ diese Lücken bestehen,
kann das Rechteck $r_i$ in die Platzierung $C'$ eingefügt werden. 
Nachdem alle durch Kollision betroffenen Streifen verarbeitet worden sind,
ist die Platzierung $C'$ fertig.

Dann erfolgt der Vergleich in \cref{line:vergleich}.
Wenn der Gesamtflächeninhalt der Platzierung $C'$ streng größer ist als
der Gesamtflächeninhalt der Platzierung $C$, wird die neue Platzierung vom Algorithmus
akzeptiert und gilt als die aktuelle Platzierung $C$.
Danach muss man offensichtlich alle Lücken und alle nicht gelegten Rechtecke neu bestimmen.
Die Iteratoren $itH$ bzw. $itR$ werden auf $H.begin$ bzw. auf $U_k.begin$ gesetzt. 

Man kann leicht begründen, dass das Programm auf jeden Fall anhält.
Durch den Quasibergsteigeralgorithmus wird ein lokales Maximum gefunden, das bestenfalls auch
das Optimum ist. 
Jedes lokale Maximum kann das Optimum nicht überschreiten.
Das Optimum ist auf den Flächeninhalt des großen Rechtecks $R$ beschränkt.
Wenn ein lokales Maximum erreicht wird, kann der Algorithmus keine neuen Platzierungen
annehmen. 
Wenn ein lokales Maximum erreicht wird, werden alle geeigenten Lücken und dazu jeweils alle 
passenden Rechtecke ausprobiert, aber das Programm nimmt keine der neuen Kombinationen an,
da sie keinen größeren Flächeninhalt bilden als der des lokalen Maximums.
Somit hält das Programm auf jeden Fall an.
 












